\documentclass[a4paper,10pt]{article}
\usepackage{eumat}

\begin{document}
\begin{eulernotebook}
\begin{eulercomment}
\begin{eulercomment}
\eulerheading{EMT Untuk Visualisasi Dan Komputasi Statistika}
\begin{eulercomment}
Kelompok 6\\
Nama Anggota Kelompok : \\
Ardan Andhirta (22305141045)\\
Nafisatul Iqima (22305144037)\\
Fransisca Renita Pejoresa (22305144012)\\
Oktavia Kusuma Wardhani (22305144013)\\
Dida Arkadia Ayu Jawata (22305144005)\\
Chintya Wijayanti (22305144029)\\
Bintang Mahija Aryacetta (22305144003)\\
Adib Brian Syuhada (22305144014)

Kelas : Matematika E 2022

\begin{eulercomment}
\eulerheading{Sub Topik 1: Menyimpan Data Dalam Bentuk Matriks}
\begin{eulercomment}
Array\\
Array adalah kumpulan-kumpulan variabel yang menyimpan data dengan
tipe yang sama atau data-data yang tersusun secara linear dimana di
dalamnya terdapat elemen dengan tipe yang sama.

Vektor digunakan untuk menggambarkan array angka satu dimensi. Vektor
memiliki panjang, yang merupakan jumlah elemen dalam array.

Sedangkan matriks digunakan dalam mendeskripsikan susunan bilangan dua\\
dimensi yang disusun dalam baris dan kolom. matriks memiliki ukuran,
yaitu jumlah baris dan kolom.

Hubungan antara array dan matriks adalah bahwa matriks adalah bentuk
khusus dari array. Array dapat memiliki lebih dari dua dimensi, tetapi
matriks selalu memiliki dua dimensi. Dalam pemrograman, array dan
matriks sering digunakan untuk menyimpan data dalam jumlah besar dan
memudahkan pengaksesan data tersebut.

Mari kita bahas beberapa hal terkait vektor terlebih dahulu

\end{eulercomment}
\begin{eulerprompt}
>v=shuffle(1:10)
\end{eulerprompt}
\begin{euleroutput}
  [6,  3,  1,  5,  10,  4,  9,  8,  2,  7]
\end{euleroutput}
\begin{eulerprompt}
>w=intrandom(10,12)
\end{eulerprompt}
\begin{euleroutput}
  [11,  4,  9,  3,  6,  4,  11,  3,  6,  2]
\end{euleroutput}
\begin{eulercomment}
Untuk mengurutkan angka acak 
\end{eulercomment}
\begin{eulerprompt}
>sort(v)
\end{eulerprompt}
\begin{euleroutput}
  [1,  2,  3,  4,  5,  6,  7,  8,  9,  10]
\end{euleroutput}
\begin{eulercomment}
Selanjutnya mengurutkan angka acak dengan menyederhanakan angka yang
sama
\end{eulercomment}
\begin{eulerprompt}
>unique(v)
\end{eulerprompt}
\begin{euleroutput}
  [1,  2,  3,  4,  5,  6,  7,  8,  9,  10]
\end{euleroutput}
\begin{eulercomment}
Menemukan banyaknya setiap elemen dengan bantuan interval
\end{eulercomment}
\begin{eulerprompt}
>s=intrandom(10,20)
\end{eulerprompt}
\begin{euleroutput}
  [12,  9,  15,  10,  7,  11,  11,  4,  6,  18]
\end{euleroutput}
\begin{eulerprompt}
>x=[5,10,15,20]
\end{eulerprompt}
\begin{euleroutput}
  [5,  10,  15,  20]
\end{euleroutput}
\begin{eulerprompt}
>find(x,s)
\end{eulerprompt}
\begin{euleroutput}
  [2,  1,  3,  2,  1,  2,  2,  0,  1,  3]
\end{euleroutput}
\begin{eulercomment}
Berikutnya adalah cara mencari indeks dari sebuah vektor dengan contoh
vekUntuk indeks pada EMT berbeda dengan indeks pada Phyton yang kita
pelajari sebelumnya di Algoritma dan pemrograman. Perbedaannya juka
sebelumnya untk menentukan indeks akan dimulalai dari nol namun di
mmenentukan indeks di EMT akan dimulai dari angka satu, berikut
penjelasannya
\end{eulercomment}
\begin{eulerprompt}
>indexof(w,1:10)
\end{eulerprompt}
\begin{euleroutput}
  [0,  10,  4,  2,  0,  5,  0,  0,  3,  0]
\end{euleroutput}
\begin{eulerprompt}
>x= sort(intrandom(10,12))
\end{eulerprompt}
\begin{euleroutput}
  [1,  1,  3,  6,  7,  8,  8,  12,  12,  12]
\end{euleroutput}
\begin{eulerprompt}
>indexofsorted(x,1:15)
\end{eulerprompt}
\begin{euleroutput}
  [2,  0,  3,  0,  0,  4,  5,  7,  0,  0,  0,  10,  0,  0,  0]
\end{euleroutput}
\begin{eulerprompt}
>z=intrandom(1000,10); multofsorted(sort(z),1:10), sum(%)
\end{eulerprompt}
\begin{euleroutput}
  [81,  96,  121,  101,  102,  91,  115,  98,  100,  95]
  1000
\end{euleroutput}
\begin{eulercomment}
Sampai disini pembahasan terkait dnegan vektor\\
Selanjutnya kita akan membahas  beberapa hal terkait matriks terkait

Untuk Menyimpan Data dalam bentuk Matrik

Pertama, buat sebuah variabel yang akan menampung data matrik, misal
X. Variabel ini bebas dengan syarat tidak sama dengan nama fungsi atau
konstanta yang sudah ada dalam software.

Selanjutnya,kita akan membuat matrik berordo mxn yang berisi angka
\end{eulercomment}
\begin{eulerprompt}
>X=[1,2,3,4;4,5,6,7;8,4,4,6]
\end{eulerprompt}
\begin{euleroutput}
              1             2             3             4 
              4             5             6             7 
              8             4             4             6 
\end{euleroutput}
\begin{eulerprompt}
>shortformat; A=random(3,4)
\end{eulerprompt}
\begin{euleroutput}
    0.50136   0.58172   0.02845   0.72032 
    0.27668    0.1313   0.84982   0.77608 
    0.32956   0.68574   0.42373   0.77217 
\end{euleroutput}
\begin{eulerprompt}
>shortformat; A=intrandom(5,4,20)
\end{eulerprompt}
\begin{euleroutput}
          3        18         4        18 
          7        19        12         8 
          5        17        11        10 
         15         1        20         7 
         11        13         9         2 
\end{euleroutput}
\begin{eulerprompt}
>shortformat; A=redim(1:15,4,4)
\end{eulerprompt}
\begin{euleroutput}
          1         2         3         4 
          5         6         7         8 
          9        10        11        12 
         13        14        15         0 
\end{euleroutput}
\begin{eulerprompt}
>(1:5)_2
\end{eulerprompt}
\begin{euleroutput}
          1         2         3         4         5 
          2         2         2         2         2 
\end{euleroutput}
\begin{eulerprompt}
>random(3,3)_random(2,2)
\end{eulerprompt}
\begin{euleroutput}
    0.74041   0.41322   0.25054 
    0.15543   0.10655   0.98859 
    0.48475  0.078167   0.57911 
    0.46856   0.22056         0 
    0.22837   0.47473         0 
\end{euleroutput}
\begin{eulerprompt}
>for k=1 to prod(size(A)); A\{k\}=k; end; short A
\end{eulerprompt}
\begin{euleroutput}
          1         2         3         4 
          5         6         7         8 
          9        10        11        12 
         13        14        15        16 
\end{euleroutput}
\begin{eulerprompt}
>B=zeros(size(A))
\end{eulerprompt}
\begin{euleroutput}
          0         0         0         0 
          0         0         0         0 
          0         0         0         0 
          0         0         0         0 
\end{euleroutput}
\begin{eulerprompt}
>B=ones(size(A))
\end{eulerprompt}
\begin{euleroutput}
          1         1         1         1 
          1         1         1         1 
          1         1         1         1 
          1         1         1         1 
\end{euleroutput}
\begin{eulercomment}
Berikutnya operasi penjumlahan dam pengurangan matriks
\end{eulercomment}
\begin{eulerprompt}
>shortformat; I=intrandom(3,4,10) 
\end{eulerprompt}
\begin{euleroutput}
          5         5         7         5 
          9         8         3         8 
          5         5         1         1 
\end{euleroutput}
\begin{eulerprompt}
>shortformat; J=intrandom(3,4,8)
\end{eulerprompt}
\begin{euleroutput}
          7         8         1         3 
          3         7         6         7 
          1         1         1         6 
\end{euleroutput}
\begin{eulerprompt}
>C= I-J
\end{eulerprompt}
\begin{euleroutput}
         -2        -3         6         2 
          6         1        -3         1 
          4         4         0        -5 
\end{euleroutput}
\begin{eulerprompt}
>C= I+J
\end{eulerprompt}
\begin{euleroutput}
         12        13         8         8 
         12        15         9        15 
          6         6         2         7 
\end{euleroutput}
\begin{eulercomment}
Dalam materi matriks yang pernah kita pelajari ada sebutan transpose,
Invers dan juga determinan, jika menggunakan EMt sebagai berikut
secera berurutan:
\end{eulercomment}
\begin{eulerprompt}
>T = transpose(I)
\end{eulerprompt}
\begin{euleroutput}
          5         9         5 
          5         8         5 
          7         3         1 
          5         8         1 
\end{euleroutput}
\begin{eulerprompt}
>T = I'
\end{eulerprompt}
\begin{euleroutput}
          5         9         5 
          5         8         5 
          7         3         1 
          5         8         1 
\end{euleroutput}
\begin{eulerprompt}
>K = J^(-1)
\end{eulerprompt}
\begin{euleroutput}
    0.14286     0.125         1   0.33333 
    0.33333   0.14286   0.16667   0.14286 
          1         1         1   0.16667 
\end{euleroutput}
\begin{eulerprompt}
>shortformat; L=intrandom(3,3,7)
\end{eulerprompt}
\begin{euleroutput}
          7         4         4 
          3         1         6 
          3         7         6 
\end{euleroutput}
\begin{eulerprompt}
>det(L)
\end{eulerprompt}
\begin{euleroutput}
  -180
\end{euleroutput}
\begin{eulercomment}
Selanjutnya adalah cara ekstraksi baris dan kolom, atau
sub-matriks,yang  mirip dengan R sebagai berikut:

\end{eulercomment}
\begin{eulerprompt}
>L[,2:3]
\end{eulerprompt}
\begin{euleroutput}
          4         4 
          1         6 
          7         6 
\end{euleroutput}
\begin{eulerprompt}
>shortformat; X=redim(1:20,4,5)
\end{eulerprompt}
\begin{euleroutput}
          1         2         3         4         5 
          6         7         8         9        10 
         11        12        13        14        15 
         16        17        18        19        20 
\end{euleroutput}
\begin{eulerprompt}
>function setmatrixvalue (M, i, j, v) ...
\end{eulerprompt}
\begin{eulerudf}
  loop 1 to max(length(i),length(j),length(v))
     M[i\{#\},j\{#\}] = v\{#\};
  end;
  endfunction
\end{eulerudf}
\begin{eulerprompt}
>setmatrixvalue(X,1:4,4:-1:1,0); X,
\end{eulerprompt}
\begin{euleroutput}
          1         2         3         0         5 
          6         7         0         9        10 
         11         0        13        14        15 
          0        17        18        19        20 
\end{euleroutput}
\begin{eulerprompt}
>(1:4)*(1:4)'
\end{eulerprompt}
\begin{euleroutput}
          1         2         3         4 
          2         4         6         8 
          3         6         9        12 
          4         8        12        16 
\end{euleroutput}
\begin{eulerprompt}
>a=0:10; b=a'; p=flatten(a*b); q=flatten(p-p'); ...
>u=sort(unique(q)); f=getmultiplicities(u,q); ...
>statplot(u,f,"h"):
\end{eulerprompt}
\eulerimg{25}{images/Kelompok 6_EMT4Statistika-001.png}
\begin{eulerprompt}
>getfrequencies(q,-50:10:50)
\end{eulerprompt}
\begin{euleroutput}
  [613,  814,  1088,  1404,  1904,  2389,  1431,  1109,  841,  680]
\end{euleroutput}
\begin{eulerprompt}
>plot2d(q,distribution=11):
\end{eulerprompt}
\eulerimg{25}{images/Kelompok 6_EMT4Statistika-002.png}
\begin{eulerprompt}
>\{x,y\}=histo(q,v=-55:10:55); y=y/sum(y)/differences(x);
>plot2d(x,y,>bar,style="/"):
\end{eulerprompt}
\eulerimg{25}{images/Kelompok 6_EMT4Statistika-003.png}
\begin{eulercomment}
\begin{eulercomment}
\eulerheading{Sub Topik 2 : Menghasilkan Data Acak Menggunakan Fungsi Distribusi}
\begin{eulercomment}
CAKUPAN MATERI MELIPUTI DIANTARANYA:\\
-Definisi Bilangan Acak dan Data Acak\\
-Pengertian Distribusi Diskrit dan Konsep yang Terkait\\
-Metode Menentukan Distribusi Diskrit


1. Definisi Bilangan Acak dan Data Acak

\end{eulercomment}
\begin{eulerttcomment}
      Bilangan Acak adalah bilangan yang tidak dapat diprediksi
\end{eulerttcomment}
\begin{eulercomment}
kemunculannya. Sehingga, tidak ada komputasi yang benar-benar
menghasilkan deret bilangan acak secara sempurna.\\
\end{eulercomment}
\begin{eulerttcomment}
      Bilangan acak sendiri dapat dibangkitkan dengan pola tertentu
\end{eulerttcomment}
\begin{eulercomment}
yang dinamakan dengan distribusi, dengan catatan mengikuti fungsi
distribusi yang ditentukan.\\
\end{eulercomment}
\begin{eulerttcomment}
      Data acak merupakan hasil dari suatu percobaan acak. Sedangkan
\end{eulerttcomment}
\begin{eulercomment}
percobaan acak adalah suatu proses yang dilakukan sedemikian rupa
sehingga hasilnya tidak dapat ditentukan dengan pasti sebelum
percobaan tersebut selesai dilakukan

contoh :
\end{eulercomment}
\begin{eulerprompt}
>intrandom(1,10,10)
\end{eulerprompt}
\begin{euleroutput}
  [2,  4,  6,  7,  3,  3,  2,  9,  10,  2]
\end{euleroutput}
\begin{eulercomment}
\end{eulercomment}
\eulersubheading{}
\begin{eulercomment}
2. Pengertian Distribusi Diskrit dan Konsep yang Terkait

\end{eulercomment}
\begin{eulerttcomment}
       Distribusi diskrit dalam statistika adalah distribusi data yang
\end{eulerttcomment}
\begin{eulercomment}
memiliki nilai-nilai yang terpisah dan dapat dihitung. Contohnya\\
adalah jumlah anak dalam sebuah keluarga, jumlah mata dadu yang\\
muncul, atau jumlah pelanggan yang datang ke sebuah toko.\\
\end{eulercomment}
\begin{eulerttcomment}
       Distribusi diskrit merujuk pada distribusi
\end{eulerttcomment}
\begin{eulercomment}
probabilitas yang melibatkan variabel acak diskrit. Variabel acak\\
diskrit adalah variabel acak yang hanya dapat mengambil nilai-nilai\\
terpisah, bukan nilai-nilai kontinu seperti pada variabel acak\\
kontinu. Distribusi diskrit memberikan probabilitas masing-masing\\
nilai yang mungkin dari variabel acak tersebut. Berikut adalah
beberapa konsep kunci yang terkait dengan distribusi diskrit dalam\\
statistika: \\
1. Fungsi Probabilitas Diskrit (Probability Mass Function - PMF):\\
-Fungsi probabilitas diskrit, atau PMF, memberikan probabilitas bahwa\\
variabel acak diskrit akan mengambil nilai tertentu.\\
PMF umumnya dilambangkan dengan P(X=x), di mana X adalah variabel acak\\
dan x adalah nilai yang mungkin dari variabel tersebut.\\
\end{eulercomment}
\begin{eulerttcomment}
 
\end{eulerttcomment}
\begin{eulercomment}
2. Ruang Sampel (Sample Space):\\
-Ruang sampel adalah himpunan semua hasil mungkin dari suatu percobaan\\
acak yang dapat diukur.\\
-Setiap elemen dalam ruang sampel merupakan hasil yang mungkin dari\\
variabel acak.\\
\end{eulercomment}
\begin{eulerttcomment}
 
\end{eulerttcomment}
\begin{eulercomment}
3. Hukum Probabilitas untuk Distribusi Diskrit:\\
Probabilitas suatu kejadian adalah bilangan yang berada dalam rentang\\
0 hingga 1, atau 0 \textless{}= P(A)\textless{}=1 untuk setiap kejadian A.\\
Probabilitas total dari semua hasil dalam ruang sampel adalah 1, atau\\
P (S)= 1, di mana S adalah ruang sampel.\\
\end{eulercomment}
\begin{eulerttcomment}
 
\end{eulerttcomment}
\begin{eulercomment}
4. Fungsi Distribusi Kumulatif (Cumulative Distribution Function -\\
CDF):\\
-Fungsi distribusi kumulatif memberikan probabilitas bahwa variabel\\
acak diskrit kurang dari atau sama dengan nilai tertentu.\\
-Notasi matematisnya sering kali disimbolkan sebagai F(x)-P(X\textless{}=x)\\
\end{eulercomment}
\begin{eulerttcomment}
 
\end{eulerttcomment}
\begin{eulercomment}
5. Harapan (Expectation) dan Varians:\\
-Harapan atau nilai rata-rata (E(X)) dari distribusi diskrit adalah\\
jumlah tertimbang dari nilai-nilai mungkin berdasarkan probabilitas\\
masing-masing nilai.\\
-Varians Var(X)) mengukur sejauh mana nilai-nilai distribusi tersebar\\
dari nilai rata-ratanya.

\end{eulercomment}
\begin{eulerprompt}
> 
\end{eulerprompt}
\eulersubheading{}
\begin{eulercomment}
3. Metode Menentukan Distribusi Diskrit

\end{eulercomment}
\begin{eulerttcomment}
      Untuk menentukan distribusi diskrit sendiri, dapat menggunakan
\end{eulerttcomment}
\begin{eulercomment}
metode berikut. Pertama kita mengatur fungsi distribusi, fungsi
distribusi adalah fungsi yang menggambarkan kemungkinan suatu variabel
acak untuk memiliki nilai tertentu atau dalam rentang waktu tertentu.\\
Langkah mengatur fungsi distribusi:\\
-Menentukan jenis var acak yg akan diteliti, apakah diskrit atau\\
kontinu\\
-Menentukan parameter-parameter yang berkaitan dengan fungsi\\
distribusi, spt probabilitas\\
-Menentukan bentuk fungsi distribusi yg sesuai dg variabel acak dan\\
parameter yg sudah ditentukan
\end{eulercomment}
\begin{eulerprompt}
>wd = 0|((1:6)+[-0.01,0.01,0,0,0,0])/5
\end{eulerprompt}
\begin{euleroutput}
  [0,  0.198,  0.402,  0.6,  0.8,  1,  1.2]
\end{euleroutput}
\begin{eulercomment}
Artinya dengan probabilitas wd[i+1]-wd[i] kita menghasilkan nilai acak
i.

Ini hampir merupakan distribusi yang seragam. Mari kita tentukan
generator angka acak untuk ini. Fungsi find(v,x) menemukan nilai x
dalam vektor v. Fungsi ini juga berlaku untuk vektor x.
\end{eulercomment}
\begin{eulerprompt}
>function wrongdice (n,m) := find(wd,random(n,m))
\end{eulerprompt}
\begin{eulercomment}
Kesalahannya sangat halus sehingga melihatnya hanya dengan iterasi
yang sangat banyak.
\end{eulercomment}
\begin{eulerprompt}
>columnsplot(getmultiplicities(1:6,wrongdice(1,1000000))):
\end{eulerprompt}
\eulerimg{25}{images/Kelompok 6_EMT4Statistika-004.png}
\begin{eulercomment}
Berikut adalah fungsi sederhana untuk memeriksa distribusi seragam
dari nilai 1...K dalam v. menerima hasilnya, jika untuk semua
frekuensi

\end{eulercomment}
\begin{eulerformula}
\[
\left|f_i-\frac{1}{K}\right| < \frac{\delta}{\sqrt{n}}.
\]
\end{eulerformula}
\begin{eulerprompt}
>function checkrandom (v, delta=1) 
\end{eulerprompt}
\begin{eulerudf}
  K=max(v); n=cols(v);
  fr=getfrequencies(v,1:K);
  return max(fr/n-1/K)<delta/sqrt(n);
  endfunction 
\end{eulerudf}
\begin{eulercomment}
Memang fungsi menolak distribusi seragam.
\end{eulercomment}
\begin{eulerprompt}
>checkrandom(wrongdice(1,1000000)) 
\end{eulerprompt}
\begin{euleroutput}
  0
\end{euleroutput}
\begin{eulercomment}
Dan itu menerima generator acak bawaan.
\end{eulercomment}
\begin{eulerprompt}
>checkrandom(intrandom(1,1000000,6))
\end{eulerprompt}
\begin{euleroutput}
  1
\end{euleroutput}
\begin{eulercomment}
Kita dapat menghitung distribusi binomial. Pertama ada binomialsum(),
yang mengembalikan probabilitas i atau kurang hit dari n percobaan.
\end{eulercomment}
\begin{eulerprompt}
>bindis(410,1000,0.4)
\end{eulerprompt}
\begin{euleroutput}
  0.7514
\end{euleroutput}
\begin{eulercomment}
Fungsi Beta terbalik digunakan untuk menghitung interval kepercayaan
Clopper-Pearson untuk parameter p. Level default adalah alfa.

Arti interval ini adalah jika p berada di luar interval, hasil
pengamatan 410 dalam 1000 jarang terjadi.
\end{eulercomment}
\begin{eulerprompt}
>clopperpearson(410,1000)
\end{eulerprompt}
\begin{euleroutput}
  [0.37932,  0.44121]
\end{euleroutput}
\begin{eulercomment}
Perintah berikut adalah cara langsung untuk mendapatkan hasil di atas.\\
Tapi untuk n besar, penjumlahan langsungnya tidak akurat dan lambat.
\end{eulercomment}
\begin{eulerprompt}
>p=0.4; i=0:410; n=1000; sum(bin(n,i)*p^i*(1-p)^(n-i))
\end{eulerprompt}
\begin{euleroutput}
  0.7514
\end{euleroutput}
\begin{eulercomment}
invbinsum() menghitung kebalikan dari binomialsum().
\end{eulercomment}
\begin{eulerprompt}
>2*hypergeomsum(1,5,13,26)
\end{eulerprompt}
\begin{euleroutput}
  0.32174
\end{euleroutput}
\begin{eulercomment}
Ada juga simulasi distribusi multinomial.Distribusi diskrit dalam
statistika adalah distribusi data yang memiliki nilai-nilai yang
terpisah dan dapat dihitung. Contohnya adalah jumlah anak dalam sebuah
keluarga, jumlah mata dadu yang muncul, atau jumlah pelanggan yang
datang ke sebuah toko.
\end{eulercomment}
\begin{eulerprompt}
>randmultinomial(10,1000,[0.4,0.1,0.5])
\end{eulerprompt}
\begin{euleroutput}
        385        95       520 
        407        88       505 
        388       106       506 
        381       114       505 
        413       113       474 
        388        99       513 
        396        91       513 
        399        93       508 
        390        89       521 
        402       103       495 
\end{euleroutput}
\eulersubheading{}
\begin{eulercomment}
Contoh Soal

Simulasikan 1000 data acak dengan distribusi normal dengan mean 1 dan
simpangan baku 2. Hitung rata-rata!

Jawab :\\
// Simulasi data acak dengan distribusi normal
\end{eulercomment}
\begin{eulerprompt}
>data = randnormal(1,1000,2)
\end{eulerprompt}
\begin{euleroutput}
  [3.9181,  1.9527,  -0.33732,  2.2526,  1.221,  1.5064,  1.0475,
  2.1997,  2.5056,  1.9128,  1.727,  2.1483,  2.2224,  3.6901,  1.1749,
  1.9995,  0.24513,  2.3104,  3.1337,  0.38286,  2.8783,  1.2136,
  2.0146,  -0.097474,  2.1818,  0.0020967,  3.6067,  2.2221,  1.6357,
  0.73029,  3.1022,  2.1763,  2.6671,  2.3198,  2.326,  1.5059,  1.4371,
  2.7432,  0.1283,  1.6684,  1.9932,  0.76937,  2.134,  1.7466,  1.3792,
  1.5807,  0.83863,  2.5515,  3.2569,  4.4061,  1.9229,  3.8785,
  1.5262,  2.3406,  1.8594,  2.4003,  2.8752,  2.5498,  2.2527,  2.7602,
  2.6761,  1.6431,  3.4518,  2.2219,  1.7896,  2.519,  2.2191,  2.1538,
  2.4901,  1.8535,  3.1297,  1.7501,  4.6754,  3.3252,  1.7295,  1.4201,
  2.515,  2.5155,  0.51684,  1.727,  1.4341,  3.1193,  1.0781,  1.1891,
  0.44626,  3.7094,  1.1403,  0.064038,  0.497,  2.4713,  2.0817,
  2.2472,  4.3022,  3.5434,  2.9278,  2.5287,  1.1453,  1.9166,  2.1864,
  1.8073,  2.3455,  0.32874,  2.8625,  1.8259,  1.6132,  1.9987,
  2.8166,  4.1364,  1.2717,  3.2232,  1.1259,  1.2758,  1.5482,  3.7335,
  1.509,  3.0431,  0.60999,  1.0187,  1.8762,  1.9697,  1.6101,  1.8664,
  2.1309,  2.2315,  2.0468,  1.559,  3.853,  1.89,  1.5104,  2.1613,
  1.5506,  5.3184,  2.3794,  2.3148,  2.5744,  1.7659,  2.6473,
  0.66693,  2.6216,  4.014,  3.6484,  -0.083378,  3.5684,  2.7952,
  2.1936,  2.381,  1.8078,  0.78504,  3.4552,  1.2495,  0.88303,
   ... ]
\end{euleroutput}
\begin{eulerprompt}
>mean(data)
\end{eulerprompt}
\begin{euleroutput}
  1.9987
\end{euleroutput}
\begin{eulercomment}
\begin{eulercomment}
\eulerheading{Sub Topik 3: Membaca Data Yang Tersimpan }
\begin{eulercomment}
Membaca Data yang Tersimpan di dalam Berkas dengan Berbagai
Format(teks biasa, CSV) untuk di analisis lebih lanjut

membaca data dari teks biasa pertama-tama, kita akan mencoba membaca
data dengan teks biasa yang terdapat pada contoh yang telah di berikan
pada besmart yaitu dengan menuliskan fungsi "printfile(nama file teks
biasa, berapa baris yang akan di print)"

seperti di bawah kita akan print tabel pada data yang terdapat pada
buku online "Einführung in die Statistik mit R" oleh A. Handl.
\end{eulercomment}
\begin{eulerprompt}
>printfile("table.dat",4)
\end{eulerprompt}
\begin{euleroutput}
  Person Sex Age Titanic Evaluation Tip Problem
  1 m 30 n . 1.80 n
  2 f 23 y g 1.80 n
  3 f 26 y g 1.80 y
\end{euleroutput}
\begin{eulercomment}
pada kasus ini saya telah menunjukan terdapat 4 baris yang telah di
print pada emt yang berisi angka dan token(string)

kali ini kita akan membaca tabel dengan lebih mudah atau dengan bahasa
kita sendiri. Untuk ini, kita mendefinisikan set token. Fungsi
strtokens () mendapatkan vektor string token dari string tertentu.
\end{eulercomment}
\begin{eulerprompt}
>mf:=["m","f"]; yn:=["y","n"]; ev:=strtokens("g vg m b vb");
\end{eulerprompt}
\begin{eulercomment}
sekarang kita dapat membacanya dengan cara kita sendiri

Argumen tok2, tok4, dll. Adalah definisi dari kolom tabel. Argumen\\
ini tidak ada dalam daftar parameter readtable (), jadi Anda perlu\\
memberinya ": =" untuk mendefinisikannya.
\end{eulercomment}
\begin{eulerprompt}
>\{MT,hd\}=readtable("table.dat",tok2:=mf,tok4:=yn,tok5:=ev,tok7:=yn);
>load over statistics;
\end{eulerprompt}
\begin{eulercomment}
lalu kita akan print tabel sesuai dengan tabel awal namun dengan
bentuk tabel yang berbeda
\end{eulercomment}
\begin{eulerprompt}
>writetable(MT[1:6],labc=hd,wc=5,tok2:=mf,tok4:=yn,tok5:=ev,tok7:=yn);
\end{eulerprompt}
\begin{euleroutput}
   Person  Sex  Age Titanic Evaluation  Tip Problem
        1    m   30       n          .  1.8       n
        2    f   23       y          g  1.8       n
        3    f   26       y          g  1.8       y
        4    m   33       n          .  2.8       n
        5    m   37       n          .  1.8       n
        6    m   28       y          g  2.8       y
\end{euleroutput}
\begin{eulercomment}
Titik "." mewakili nilai-nilai yang tidak tersedia.

Jika kita tidak ingin menentukan token untuk terjemahan terlebih
dahulu, kita hanya perlu menentukan, kolom mana yang berisi token dan
bukan angka.
\end{eulercomment}
\begin{eulerprompt}
>ctok=[2,4,5,7]; \{MT,hd,tok\}=readtable("table.dat",ctok=ctok);  
\end{eulerprompt}
\begin{eulercomment}
Fungsi readtable () sekarang mengembalikan satu set token.
\end{eulercomment}
\begin{eulerprompt}
>tok
\end{eulerprompt}
\begin{euleroutput}
  m
  n
  f
  y
  g
  vg
\end{euleroutput}
\begin{eulercomment}
Tabel berisi entri dari file dengan token yang diterjemahkan menjadi
angka.

String khusus NA = "." diartikan sebagai "Tidak Tersedia", dan
mendapatkan NAN (bukan angka) di tabel. Terjemahan ini dapat diubah
dengan parameter NA, dan NAval.
\end{eulercomment}
\begin{eulerprompt}
>MT[1]
\end{eulerprompt}
\begin{euleroutput}
  [1,  1,  30,  2,  NAN,  1.8,  2]
\end{euleroutput}
\begin{eulercomment}
Berikut adalah isi tabel dengan bilangan yang belum diterjemahkan.
\end{eulercomment}
\begin{eulerprompt}
>writetable(MT[1:6],wc=5)
\end{eulerprompt}
\begin{euleroutput}
      1    1   30    2    .  1.8    2
      2    3   23    4    5  1.8    2
      3    3   26    4    5  1.8    4
      4    1   33    2    .  2.8    2
      5    1   37    2    .  1.8    2
      6    1   28    4    5  2.8    4
\end{euleroutput}
\begin{eulercomment}
Untuk kenyamanan, Anda bisa memasukkan keluaran readtable () ke dalam\\
daftar.
\end{eulercomment}
\begin{eulerprompt}
>Table=\{\{readtable("table.dat",ctok=ctok)\}\}; 
\end{eulerprompt}
\begin{eulercomment}
Dengan menggunakan kolom token yang sama dan token dibaca dari file,
kita dapat mencetak tabel. Kita dapat menentukan ctok, tok, dll. Atau
menggunakan Tabel daftar.
\end{eulercomment}
\begin{eulerprompt}
>writetable(Table,ctok=ctok,wc=5);
\end{eulerprompt}
\begin{euleroutput}
   Person  Sex  Age Titanic Evaluation  Tip Problem
        1    m   30       n          .  1.8       n
        2    f   23       y          g  1.8       n
        3    f   26       y          g  1.8       y
        4    m   33       n          .  2.8       n
        5    m   37       n          .  1.8       n
        6    m   28       y          g  2.8       y
        7    f   31       y         vg  2.8       n
        8    m   23       n          .  0.8       n
        9    f   24       y         vg  1.8       y
       10    m   26       n          .  1.8       n
       11    f   23       y         vg  1.8       y
       12    m   32       y          g  1.8       n
       13    m   29       y         vg  1.8       y
       14    f   25       y          g  1.8       y
       15    f   31       y          g  0.8       n
       16    m   26       y          g  2.8       n
       17    m   37       n          .  3.8       n
       18    m   38       y          g    .       n
       19    f   29       n          .  3.8       n
       20    f   28       y         vg  1.8       n
       21    f   28       y          m  2.8       y
       22    f   28       y         vg  1.8       y
       23    f   38       y          g  2.8       n
       24    f   27       y          m  1.8       y
       25    m   27       n          .  2.8       y
\end{euleroutput}
\begin{eulercomment}
tabel sudah dapat di analisis lebih lanjut
\end{eulercomment}
\eulerheading{membaca data dari CSV}
\begin{eulercomment}
pertama tama kita download file csv yang telah di sediakan di besmart,
setelah itu kita jadi satukan dalam 1 folder dengan file emt kita.
lalu masukan file tersebut dengan definisi file="nama file csv"
\end{eulercomment}
\begin{eulerprompt}
>file="test.csv";  ...
>M=random(3,3); writematrix(M,file)
\end{eulerprompt}
\begin{eulercomment}
M mendefinisikan sebagai matrix\\
random(n,m) mendefinisikan matrix dengan variabel acak yang akan di
keluarakan\\
writematrix digunakan untuk menuliskan matriks yang ada

lalu kita print datanya dengan
\end{eulercomment}
\begin{eulerprompt}
>printfile(file)
\end{eulerprompt}
\begin{euleroutput}
  0.4191983703672241,0.4504034185243261,0.8530870403686531
  0.7175631662797505,0.6559204322999515,0.883095901954524
  0.07277738544441656,0.8168773103666358,0.5703787256414963
  
\end{euleroutput}
\begin{eulercomment}
titik desimal pada data tersebut dapat di jadikan pada format EMT
dengan cara menggunakan readmatrix()
\end{eulercomment}
\begin{eulerprompt}
>readmatrix(file)
\end{eulerprompt}
\begin{euleroutput}
     0.4192    0.4504   0.85309 
    0.71756   0.65592    0.8831 
   0.072777   0.81688   0.57038 
\end{euleroutput}
\begin{eulerprompt}
> 
\end{eulerprompt}
\begin{eulercomment}
Di Excel atau spreadsheet serupa, Anda dapat mengekspor matriks
sebagai CSV (nilai dipisahkan koma). Di Excel 2007, gunakan "simpan
sebagai" dan "format lain", lalu pilih "CSV". Pastikan, tabel saat ini
hanya berisi data yang ingin Anda ekspor.

Berikut ini contohnya.
\end{eulercomment}
\begin{eulerprompt}
>printfile("excel-data.csv")
\end{eulerprompt}
\begin{euleroutput}
  0;1000;1000
  1;1051,271096;1072,508181
  2;1105,170918;1150,273799
  3;1161,834243;1233,67806
  4;1221,402758;1323,129812
  5;1284,025417;1419,067549
  6;1349,858808;1521,961556
  7;1419,067549;1632,31622
  8;1491,824698;1750,6725
  9;1568,312185;1877,610579
  10;1648,721271;2013,752707
\end{euleroutput}
\begin{eulercomment}
Seperti yang Anda lihat, sistem Jerman saya menggunakan titik koma
sebagai pemisah dan koma desimal. Anda dapat mengubahnya di pengaturan
sistem atau di Excel, tetapi tidak perlu membaca matriks ke EMT.

Cara termudah untuk membaca ini ke dalam Euler adalah readmatrix ().
Semua koma diganti dengan titik dengan parameter\textgreater{} koma. Untuk CSV
bahasa Inggris, cukup abaikan parameter ini.
\end{eulercomment}
\begin{eulerprompt}
>M=readmatrix("excel-data.csv",>comma)
\end{eulerprompt}
\begin{euleroutput}
          0      1000      1000 
          1    1051.3    1072.5 
          2    1105.2    1150.3 
          3    1161.8    1233.7 
          4    1221.4    1323.1 
          5      1284    1419.1 
          6    1349.9      1522 
          7    1419.1    1632.3 
          8    1491.8    1750.7 
          9    1568.3    1877.6 
         10    1648.7    2013.8 
\end{euleroutput}
\begin{eulercomment}
data siap di analisis lebih lanjut
\end{eulercomment}
\begin{eulerprompt}
>reset;
\end{eulerprompt}
\eulerheading{Latihan}
\eulerheading{nomer 1}
\begin{eulerprompt}
>file="sample.csv"
\end{eulerprompt}
\begin{euleroutput}
  sample.csv
\end{euleroutput}
\begin{eulerprompt}
>printfile(file,7)
\end{eulerprompt}
\begin{euleroutput}
  female,read,write,math,hon,femalexmath
  0,57,52,41,0,0
  1,68,59,53,0,53
  0,44,33,54,0,0
  0,63,44,47,0,0
  0,47,52,57,0,0
  0,44,52,51,0,0
\end{euleroutput}
\begin{eulerprompt}
>ctok=[1]; \{MT,hd,tok\}=readtable("sample.csv",ctok=ctok);
>tok
\end{eulerprompt}
\begin{euleroutput}
  0
  1
\end{euleroutput}
\begin{eulerprompt}
>MT[2]
\end{eulerprompt}
\begin{euleroutput}
  [2,  68,  59,  53,  0,  53]
\end{euleroutput}
\begin{eulerprompt}
>writetable(MT[1:6],wc=5)
\end{eulerprompt}
\begin{euleroutput}
      1   57   52   41    0    0
      2   68   59   53    0   53
      1   44   33   54    0    0
      1   63   44   47    0    0
      1   47   52   57    0    0
      1   44   52   51    0    0
\end{euleroutput}
\begin{eulerprompt}
>Table=\{\{readtable(file,ctok=ctok)\}\}; 
>writetable(Table,ctok=ctok,wc=10);
\end{eulerprompt}
\begin{euleroutput}
      female      read     write      math       hon femalexmath
           0        57        52        41         0           0
           1        68        59        53         0          53
           0        44        33        54         0           0
           0        63        44        47         0           0
           0        47        52        57         0           0
           0        44        52        51         0           0
           0        50        59        42         0           0
           0        34        46        45         0           0
           0        63        57        54         0           0
           0        57        55        52         0           0
           0        60        46        51         0           0
           0        57        65        51         1           0
           0        73        60        71         0           0
           0        54        63        57         1           0
           0        45        57        50         0           0
           0        42        49        43         0           0
           0        47        52        51         0           0
           0        57        57        60         0           0
           0        68        65        62         1           0
           0        55        39        57         0           0
           0        63        49        35         0           0
           0        63        63        75         1           0
           0        50        40        45         0           0
           0        60        52        57         0           0
           0        37        44        45         0           0
           0        34        37        46         0           0
           0        65        65        66         1           0
           0        47        57        57         0           0
           0        44        38        49         0           0
           0        52        44        49         0           0
           0        42        31        57         0           0
           0        76        52        64         0           0
           0        65        67        63         1           0
           0        42        41        57         0           0
           0        52        59        50         0           0
           0        60        65        58         1           0
           0        68        54        75         0           0
           0        65        62        68         1           0
           0        47        31        44         0           0
           0        39        31        40         0           0
           0        47        47        41         0           0
           0        55        59        62         0           0
           0        52        54        57         0           0
           0        42        41        43         0           0
           0        65        65        48         1           0
           0        55        59        63         0           0
           0        50        40        39         0           0
           0        65        59        70         0           0
           0        47        59        63         0           0
           0        57        54        59         0           0
           0        53        61        61         1           0
           0        39        33        38         0           0
           0        44        44        61         0           0
           0        63        59        49         0           0
           0        73        62        73         1           0
           0        39        39        44         0           0
           0        37        37        42         0           0
           0        42        39        39         0           0
           0        63        57        55         0           0
           0        48        49        52         0           0
           0        50        46        45         0           0
           0        47        62        61         1           0
           0        44        44        39         0           0
           0        34        33        41         0           0
           0        50        42        50         0           0
           0        44        41        40         0           0
           0        60        54        60         0           0
           0        47        39        47         0           0
           0        63        43        59         0           0
           0        50        33        49         0           0
           0        44        44        46         0           0
           0        60        54        58         0           0
           0        73        67        71         1           0
           0        68        59        58         0           0
           0        55        45        46         0           0
           0        47        40        43         0           0
           0        55        61        54         1           0
           0        68        59        56         0           0
           0        31        36        46         0           0
           0        47        41        54         0           0
           0        63        59        57         0           0
           0        36        49        54         0           0
           0        68        59        71         0           0
           0        63        65        48         1           0
           0        55        41        40         0           0
           0        55        62        64         1           0
           0        52        41        51         0           0
           0        34        49        39         0           0
           0        50        31        40         0           0
           0        55        49        61         0           0
           0        52        62        66         1           0
           0        63        49        49         0           0
           1        68        62        65         1          65
           1        39        44        52         0          52
           1        44        44        46         0          46
           1        50        62        61         1          61
           1        71        65        72         1          72
           1        63        65        71         1          71
           1        34        44        40         0          40
           1        63        63        69         1          69
           1        68        60        64         0          64
           1        47        59        56         0          56
           1        47        46        49         0          49
           1        63        52        54         0          54
           1        52        59        53         0          53
           1        55        54        66         0          66
           1        60        62        67         1          67
           1        35        35        40         0          40
           1        47        54        46         0          46
           1        71        65        69         1          69
           1        57        52        40         0          40
           1        44        50        41         0          41
           1        65        59        57         0          57
           1        68        65        58         1          58
           1        73        61        57         1          57
           1        36        44        37         0          37
           1        43        54        55         0          55
           1        73        67        62         1          62
           1        52        57        64         0          64
           1        41        47        40         0          40
           1        60        54        50         0          50
           1        50        52        46         0          46
           1        50        52        53         0          53
           1        47        46        52         0          52
           1        47        62        45         1          45
           1        55        57        56         0          56
           1        50        41        45         0          45
           1        39        53        54         0          54
           1        50        49        56         0          56
           1        34        35        41         0          41
           1        57        59        54         0          54
           1        57        65        72         1          72
           1        68        62        56         1          56
           1        42        54        47         0          47
           1        61        59        49         0          49
           1        76        63        60         1          60
           1        47        59        54         0          54
           1        46        52        55         0          55
           1        39        41        33         0          33
           1        52        49        49         0          49
           1        28        46        43         0          43
           1        42        54        50         0          50
           1        47        42        52         0          52
           1        47        57        48         0          48
           1        52        59        58         0          58
           1        47        52        43         0          43
           1        50        62        41         1          41
           1        44        52        43         0          43
           1        47        41        46         0          46
           1        45        55        44         0          44
           1        47        37        43         0          43
           1        65        54        61         0          61
           1        43        57        40         0          40
           1        47        54        49         0          49
           1        57        62        56         1          56
           1        68        59        61         0          61
           1        52        55        50         0          50
           1        42        57        51         0          51
           1        42        39        42         0          42
           1        66        67        67         1          67
           1        47        62        53         1          53
           1        57        50        50         0          50
           1        47        61        51         1          51
           1        57        62        72         1          72
           1        52        59        48         0          48
           1        44        44        40         0          40
           1        50        59        53         0          53
           1        39        54        39         0          39
           1        57        62        63         1          63
           1        57        60        51         0          51
           1        42        57        45         0          45
           1        47        46        39         0          39
           1        42        36        42         0          42
           1        60        59        62         0          62
           1        44        49        44         0          44
           1        63        60        65         0          65
           1        65        67        63         1          63
           1        39        54        54         0          54
           1        50        52        45         0          45
           1        52        65        60         1          60
           1        60        62        49         1          49
           1        44        49        48         0          48
           1        52        67        57         1          57
           1        55        65        55         1          55
           1        50        67        66         1          66
           1        65        65        64         1          64
           1        52        54        55         0          55
           1        47        44        42         0          42
           1        63        62        56         1          56
           1        50        46        53         0          53
           1        42        54        41         0          41
           1        36        57        42         0          42
           1        50        52        53         0          53
           1        41        59        42         0          42
           1        47        65        60         1          60
           1        55        59        52         0          52
           1        42        46        38         0          38
           1        57        41        57         0          57
           1        55        62        58         1          58
           1        63        65        65         1          65
\end{euleroutput}
\begin{eulerprompt}
>reset;
\end{eulerprompt}
\eulersubheading{}
\begin{eulercomment}
nomer 2
\end{eulercomment}
\begin{eulerprompt}
>file="test.dat"
\end{eulerprompt}
\begin{euleroutput}
  test.dat
\end{euleroutput}
\begin{eulerprompt}
>printfile(file,7)
\end{eulerprompt}
\begin{euleroutput}
  A,B,C
  0.5403081010683386,0.9548801091335757,0.1530175968103556
  0.8596893833361555,0.975812107764198,0.0443954899784255
  0.6000027218834643,0.6869600249415273,0.7555738981923933
  
\end{euleroutput}
\begin{eulerprompt}
>mf:=["m","f"];
>\{MT,hd\}=readtable("table1.dat",tok2:=mf);
>load over statistics;
>writetable(MT[1:6],labc=hd,wc=5,tok2:=mf);
\end{eulerprompt}
\begin{euleroutput}
   Person  Sex  Age Mother Father Siblings
        1    m   29     58     61        1
        2    f   26     53     54        2
        3    m   24     49     55        1
        4    f   25     56     63        3
        5    f   25     49     53        0
        6    f   23     55     55        2
\end{euleroutput}
\begin{eulerprompt}
>reset;
\end{eulerprompt}
\begin{eulercomment}
\begin{eulercomment}
\eulerheading{Sub Topik 4: Membaca data dari internet }
\begin{eulercomment}
Membaca data dari internet

Situs web atau file dari URL dapat dibuka dengan menggunakan EMT dan
dapat dibaca baris demi baris.

Berikut contoh penggunaan EMT untuk membuka url untuk mengetahui versi
dari EMT
\end{eulercomment}
\begin{eulerprompt}
>function readversion () ...
\end{eulerprompt}
\begin{eulerudf}
  urlopen("http://www.euler-math-toolbox.de/Programs/Changes.html"); //membuka url
  repeat //loop yang berlangsung sampai akhir file url
  until urleof();
  s=urlgetline();  //membaca baris teks
  k=strfind(s,"Version",1);  //mencari substring"Version". jika ditemukan akan disimpan di k
  if k>0 then substring(s,k,strfind(s,"<",k)-1), break; endif; //berhenti sebelum <
  end;
  urlclose();
  endfunction
\end{eulerudf}
\begin{eulerprompt}
> readversion
\end{eulerprompt}
\begin{euleroutput}
  Version 2022-05-18
\end{euleroutput}
\begin{eulerprompt}
>function readdataurl () ...
\end{eulerprompt}
\begin{eulerudf}
  urlopen("https://kumparan.com/berita-terkini/3-contoh-soal-desil-data-tunggal-beserta-kunci-jawabannya-216tqjqjTvn/1");
  repeat
  until urleof();
  s=urlgetline();
  k=strfind(s,"Tentukan persentil",1);
  if k>0 then substring(s,k,strfind(s,".",k)-1), break; endif;
  end;
  urlclose();
  endfunction
\end{eulerudf}
\begin{eulercomment}
Selanjutnya kita mencoba dengan cara yang sama untuk mengambil soal
dari website yang ada di internet
\end{eulercomment}
\begin{eulerprompt}
>readdataurl
\end{eulerprompt}
\eulerheading{Sub Topik 5: Perhitungan terkait analisis data statistika deskriptif}
\begin{eulercomment}
Rata-rata, simpangan baku, jangkauan, modus, ukuran data,varians dan
median.\\
\begin{eulercomment}
\eulerheading{Analisis data statistika deskriptif}
\begin{eulercomment}
Statistika deskriptif adalah bidang ilmu statistika yang mempelajari
cara-cara untuk pengumpulan, penyusunan, dan penyajian data sehingga
memberikan informasi yang berguna. Perlu diketahui juga bahwa
statistika deskriptif memberikan informasi hanya mengenai data yang
dipunyai dan sama sekali tidak menarik inferensia atau kesimpulan
apapun tentang gugus data induknya yang lebih besar.

Dalam praktiknya,analisis data statistika deskriptif bisa dilakukan
dengan menerapkan sejumlah metode statistik, seperti :

\end{eulercomment}
\eulersubheading{1. Mencari rata rata/mean}
\begin{eulercomment}
Metode pertama yang digunakan untuk melakukan analisis statistika
adalah mean atau sering disebut rata-rata. Saat akan menghitung
rata-rata, kita bisa melakukan dengan cara menambahkan daftar angka
kemudian membagi angka tersebut dengan jumlah item dalam daftar.
Metode ini memungkinkan penentuan tren keseluruhan dari kumpulan data
dan mampu mendapatkan tampilan data yang cepat dan ringkas. Manfaat
dari metode ini juga termasuk perhitungan yang sederhana dan cepat.

\end{eulercomment}
\eulersubheading{a. Rata-rata hitung data tunggal}
\begin{eulercomment}
Misalkan\\
\end{eulercomment}
\begin{eulerformula}
\[
x_1 , x_2 , x_3 ,..., x_n
\]
\end{eulerformula}
\begin{eulercomment}
adalah data yang dikumpulkan dari suatu sampel atau populasi maka
rata-rata hitung untuk sampel disimbolkan dengan\\
\end{eulercomment}
\begin{eulerformula}
\[
\bar{x}
\]
\end{eulerformula}
\begin{eulercomment}
dan rata-rata hitung untuk populasi disimbolkan dengan\\
\end{eulercomment}
\begin{eulerformula}
\[
\mu
\]
\end{eulerformula}
\begin{eulercomment}
Sehingga, untuk mencari rata-rata hitung data tunggal terdapat 2 jenis
rumus sebagai berikut :\\
1. Rata-rata hitung sampel\\
\end{eulercomment}
\begin{eulerformula}
\[
\bar{x}=\frac{\sum_{i=1}^{n} x_i}{n}
\]
\end{eulerformula}
\begin{eulercomment}
2. Rata-rata hitung populasi,\\
\end{eulercomment}
\begin{eulerformula}
\[
\mu=\frac{\sum_{i=1}^{n} X_i}{n}
\]
\end{eulerformula}
\begin{eulercomment}
Untuk menghitung rata-rata data tunggal dengan EMT, kita dapat
menggunakan sintaks

\textgreater{} mean ([data])

Contoh Soal:\\
1. Diketahui data usia(dalam tahun) penduduk suatu daerah adalah
sebagai berikut:\\
60,70,66,75,77,68,45,30,15,71,69,84,13\\
hitunglah rata-rata usia penduduk tersebut.\\
Jawab :
\end{eulercomment}
\begin{eulerprompt}
>mean([60,70,66,75,77,68,45,30,15,71,69,84,13])
\end{eulerprompt}
\begin{euleroutput}
  57.1538461538
\end{euleroutput}
\begin{eulercomment}
Jadi, rata rata data tersebut adalah 57.1538461538

2. Nilai ulangan matematika dari 10 siswa adalah 80, 88, 70, 60, 90,
75, 92, 78, 67, 90. Tentukan rata-rata dari data tersebut!\\
Jawab :
\end{eulercomment}
\begin{eulerprompt}
>mean([80, 88, 70, 60, 90, 75, 92, 78, 67, 90])
\end{eulerprompt}
\begin{euleroutput}
  79
\end{euleroutput}
\begin{eulercomment}
Jadi, rata-rata dari data tersebut yaitu 79

\end{eulercomment}
\eulersubheading{b. Rata-rata data tabel distribusi}
\begin{eulercomment}
Jika diberikan data\\
\end{eulercomment}
\begin{eulerformula}
\[
x_1,x_2,...,x_n
\]
\end{eulerformula}
\begin{eulercomment}
yang memiliki frekuensi berturut- turut\\
\end{eulercomment}
\begin{eulerformula}
\[
f_1,f_2,...,f_n
\]
\end{eulerformula}
\begin{eulercomment}
maka, rataan hitung dari data yang disajikan dalam daftar distribusi
tersebut ditentukan dengan 2 jenis rumus sebagai berikut :

1. Rata-rata hitung sampel\\
Untuk rata-rata hitung sampel,\\
\end{eulercomment}
\begin{eulerformula}
\[
\bar{x}=\frac{\sum_{i=1}^{n} f_i x_i}{\sum_{i=1}^{n} f_i}
\]
\end{eulerformula}
\begin{eulercomment}
2.Rata-rata hitung populasi\\
Untuk rata-rata hitung populasi,\\
\end{eulercomment}
\begin{eulerformula}
\[
\mu=\frac{\sum_{i=1}^{n} f_i x_i}{\sum_{i=1}^{n} f_i}
\]
\end{eulerformula}
\begin{eulercomment}
Cara diatas adalah beberapa perhitungan untuk mencari rata-rata data
tabel distribusi menggunakan metode yang ada dalam statistika. Dengan
menggunakan EMT kita juga bisa menghitung rata-rata data tabel
distribusi dengan mudah, yaitu dengan cara berikut:\\
1. Mendeskripsikan data dan frekuensi\\
2. Menghitung rata-rata menggunakan perintah berikut :

\textgreater{} mean(data,frekuensi)

Contoh soal:\\
Diberikan data berat badan siswa kelas V SD yang memiliki jumlah siswa
sebanyak 35 orang anak. anak dengan berat 30kg terdapat 5 orang, anak
dengan berat 35kg terdapat 11 orang, anak dengan berat 40kg terdapat 4
orang, anak dengan berat 38kg terdapat 7 orang, anak dengan berat 44kg
terdapat 7 orang, dan anak dengan berat 50kg terdapat 1 orang.
Tentukan rata-rata berat siswa kelas V SD tersebut!\\
Jawab :
\end{eulercomment}
\begin{eulerprompt}
>printfile("tabel berat badan kelas V SD.dat",7); //meringkas informasi pada soal dengan membuat tabel
\end{eulerprompt}
\begin{euleroutput}
  Berat Badan(Kg) Frekuensi
        30            5
        35            11
        38            7
        40            4
        44            7
        50            1
\end{euleroutput}
\begin{eulerprompt}
>data=[30,35,38,40,44,50]//mendefinisikan data sebagai berat siswa dalam satuan kilogram
\end{eulerprompt}
\begin{euleroutput}
  [30,  35,  38,  40,  44,  50]
\end{euleroutput}
\begin{eulerprompt}
>frekuensi=[5,11,7,4,7,1]//mendefinisikan frekuensi sebagai banyak siswa
\end{eulerprompt}
\begin{euleroutput}
  [5,  11,  7,  4,  7,  1]
\end{euleroutput}
\begin{eulerprompt}
>mean(data,frekuensi) //menghitung rata-rata
\end{eulerprompt}
\begin{euleroutput}
  37.6857142857
\end{euleroutput}
\begin{eulercomment}
Jadi, rata-rata berat badan siswa SD kelas V adalah 37.6857142857
\end{eulercomment}
\eulersubheading{c. Rata-rata hitung data kelompok}
\begin{eulercomment}
Misalkan suatu data kelompok terdiri dari n kelas dengan nilai tengah
masing-masing kelas secara berturut-turut adalah\\
\end{eulercomment}
\begin{eulerformula}
\[
t_1, t_2,...,t_n
\]
\end{eulerformula}
\begin{eulercomment}
dan masing-masing frekuensinya adalah\\
\end{eulercomment}
\begin{eulerformula}
\[
f_1, f_2,..., f_n
\]
\end{eulerformula}
\begin{eulercomment}
Untuk mencari rata rata hitung data tersebut terdapat 2 jenis rumus
sebagai berikut :\\
1. Rata-rata hitung sampel\\
untuk rata-rata hitung sampel,\\
\end{eulercomment}
\begin{eulerformula}
\[
\bar{x}=\frac{\sum_{i=1}^{n} t_i f_i}{\sum_{i=1}^{n} f_i}
\]
\end{eulerformula}
\begin{eulercomment}
2. Rata-rata hitung populasi\\
untuk rata-rata hitung populasi,\\
\end{eulercomment}
\begin{eulerformula}
\[
\mu=\frac{\sum_{i=1}^{n} t_i f_i}{\sum_{i=1}^{n} f_i}
\]
\end{eulerformula}
\begin{eulercomment}
Untuk menghitung rata-rata data kelompok di EMT dapat dilakukan dengan
langkah berikut :\\
1. Menentukan tepi bawah kelas(Tb), panjang kelas(P), dan tepi atas
kelas(Ta) dengan rumus :\\
\end{eulercomment}
\begin{eulerformula}
\[
Tb=a-0,5
\]
\end{eulerformula}
\begin{eulerformula}
\[
P=(b-a)+1
\]
\end{eulerformula}
\begin{eulerformula}
\[
Ta=b+0,5
\]
\end{eulerformula}
\begin{eulercomment}
Keterangan :\\
a = batas bawah kelas\\
b = batas atas kelas


2. Membuat data menjadi bentuk tabel, dengan perintah

\textgreater{} r= tepi bawah terkecil : panjang kelas : tepi atas terbesar;\\
f=[frekuensi];\\
\textgreater{}T:r[1:jumlah kelas]' \textbar{} r[2:jumlah kelas + 1]' \textbar{}f';\\
writetable(T, labc=["tepi bawah", "tapi atas", "frekuensi"])

3. Menghitung nilai tengah kelas, dengan perintah

\textgreater{}T[,1]+T[,2]/2

4. Mengubah baris menjadi kolom

\textgreater{}t=fold(r,[0.5,0.5])

5. Menghitung rata-rata, dengan perintah

\textgreater{}mean(t,f)

Contoh soal :\\
1. Disajikan data kelompok seperti berikut :
\end{eulercomment}
\begin{eulerprompt}
>printfile("Tabel rata-rata data kelompok.dat",7)
\end{eulerprompt}
\begin{euleroutput}
        Kelas     Frekuensi
        31-40         3
        41-50         5
        51-60         10
        61-70         11
        71-80         8
        81-90         3
\end{euleroutput}
\begin{eulerprompt}
>31-0.5  //Tepi bawah terkecil
\end{eulerprompt}
\begin{euleroutput}
  30.5
\end{euleroutput}
\begin{eulerprompt}
>(40-31)+1  //Panjang kelas
\end{eulerprompt}
\begin{euleroutput}
  10
\end{euleroutput}
\begin{eulerprompt}
>90+0.5 //Tepi atas kelas
\end{eulerprompt}
\begin{euleroutput}
  90.5
\end{euleroutput}
\begin{eulerprompt}
>r=30.5:10:90.5; f=[3, 5, 10, 11, 8, 3];
>T:=r[1:6]' | r[2:7]' | f' ; writetable(T,labc=["tepi bawah", "tepi atas", "frekuensi"])
\end{eulerprompt}
\begin{euleroutput}
   tepi bawah tepi atas frekuensi
         30.5      40.5         3
         40.5      50.5         5
         50.5      60.5        10
         60.5      70.5        11
         70.5      80.5         8
         80.5      90.5         3
\end{euleroutput}
\begin{eulerprompt}
>t=(T[,1]+T[,2])/2  //menghitung nilai tengah kelas
\end{eulerprompt}
\begin{euleroutput}
           35.5 
           45.5 
           55.5 
           65.5 
           75.5 
           85.5 
\end{euleroutput}
\begin{eulerprompt}
>t=fold(r,[0.5,0.5]) // mengubah tampilan data kolom menjadi baris dan sebaliknya
\end{eulerprompt}
\begin{euleroutput}
  [35.5,  45.5,  55.5,  65.5,  75.5,  85.5]
\end{euleroutput}
\begin{eulerprompt}
>mean(t,f)
\end{eulerprompt}
\begin{euleroutput}
  61.75
\end{euleroutput}
\begin{eulercomment}
Jadi, rata-rata data kelompok tersebut adalah 61,75

2. Diberikan data kelompok berikut yang mewakili jumlah jam belajar
per minggu dari sekelompok siswa :
\end{eulercomment}
\begin{eulerprompt}
>printfile("Tabel data kelompok conso 2.dat",5)
\end{eulerprompt}
\begin{euleroutput}
      Kelompok        Frekuensi
       10-14              5
       15-19              8
       20-24              12
       25-29              6
\end{euleroutput}
\begin{eulercomment}
Hitunglah rata-rata jumlah jam belajar per minggu dari data kelompok
tersebut!\\
Jawab :
\end{eulercomment}
\begin{eulerprompt}
>10-0.5  //tepi bawah terkecil
\end{eulerprompt}
\begin{euleroutput}
  9.5
\end{euleroutput}
\begin{eulerprompt}
>(14-10)+1  //panjang kelas
\end{eulerprompt}
\begin{euleroutput}
  5
\end{euleroutput}
\begin{eulerprompt}
>29+0.5  //tepi atas terbesar
\end{eulerprompt}
\begin{euleroutput}
  29.5
\end{euleroutput}
\begin{eulerprompt}
>r=9.5:5:29.5; f=[5, 8, 12, 6];
>T:=r[1:4]' | r[2:5]' | f'; writetable(T,labc=["tepi bawah", "tepi atas", "frekuensi"])
\end{eulerprompt}
\begin{euleroutput}
   tepi bawah tepi atas frekuensi
          9.5      14.5         5
         14.5      19.5         8
         19.5      24.5        12
         24.5      29.5         6
\end{euleroutput}
\begin{eulerprompt}
>t=(T[,1]+T[,2])/2  // menghitung nilai tengah kelas
\end{eulerprompt}
\begin{euleroutput}
             12 
             17 
             22 
             27 
\end{euleroutput}
\begin{eulerprompt}
>t=fold(r,[0.5,0.5]) // mengubah tampilan data kolom menjadi baris dan sebaliknya
\end{eulerprompt}
\begin{euleroutput}
  [12,  17,  22,  27]
\end{euleroutput}
\begin{eulerprompt}
>mean(t,f)
\end{eulerprompt}
\begin{euleroutput}
  20.064516129
\end{euleroutput}
\begin{eulercomment}
Jadi, rata-rata data kelompok tersebut yaitu 20.064516129\\
\end{eulercomment}
\eulersubheading{2. Mencari median}
\begin{eulercomment}
Median (Me) adalah nilai tengah dari suatu data yang telah disusun
dari data terkecil sampai data terbesar atau sebaliknya. Selain
sebagai ukuran pemusatan data, median juga dijadikan sebagai ukuran
letak data dan dikenal sebagai kuartil 2 (Q2). Rumus perhitungan
median dibedakan untuk data tak berkelompok dan data berkelompok.

\end{eulercomment}
\eulersubheading{a. Median data tunggal}
\begin{eulercomment}
Median data tunggal adalah mengurutkan data berdasarkan nilainya,
misalkan data yang telah terurut dari data terkecil ke data terbesar
adalah\\
\end{eulercomment}
\begin{eulerformula}
\[
x_1, x_2,..., x_n
\]
\end{eulerformula}
\begin{eulerttcomment}
 untuk menentukan letak median dengan menggunakan rumus :
\end{eulerttcomment}
\begin{eulercomment}
1. Jika jumlah suatu data(n) berjumlah ganjil maka nilai mediannya
adalah sama dengan data yang memiliki nilai di urutan paling tengah
yang memiliki nomor urut k, dimana untuk menentukan nilai k dapat
dihitung menggunakan rumus:

\end{eulercomment}
\begin{eulerformula}
\[
k=\frac{n+1}2
\]
\end{eulerformula}
\begin{eulercomment}
2. Jika jumlah suatu data (n) berjumlah genap, maka untuk menghitung
mediannya dengan menggunakan rumus :\\
\end{eulercomment}
\begin{eulerformula}
\[
k=\frac{n}2
\]
\end{eulerformula}
\begin{eulerformula}
\[
Median = \frac{1}2(x_k+x_{k+1})
\]
\end{eulerformula}
\begin{eulercomment}
Diatas adalah rumus untuk mencari median secara statistika. Dengan
menggunakan EMT kita bisa menentukan median dengan menggunakan
perintah

\textgreater{} median([data])

perintah tersebut dapat berjalan dengan baik apabila data sudah
diurutkan terlebih dahulu dari data terkecil hingga terbesar.\\
Contoh soal :\\
Diketahui data hasil tes SKD calon PNS adalah sebagai berikut :\\
487, 300, 450, 500, 521, 440\\
Tentukan nilai median dari data tersebut!\\
Jawab :
\end{eulercomment}
\begin{eulerprompt}
>data=[487, 300, 450, 500, 521, 440]; //mendeskripsikan data
>urutan=sort(data)  //mengurutkan data
\end{eulerprompt}
\begin{euleroutput}
  [300,  440,  450,  487,  500,  521]
\end{euleroutput}
\begin{eulerprompt}
>median([urutan])
\end{eulerprompt}
\begin{euleroutput}
  468.5
\end{euleroutput}
\begin{eulercomment}
Jadi, nilai median dari data hasil tes SKD adalah 468.5\\
\end{eulercomment}
\eulersubheading{b. Median data kelompok}
\begin{eulercomment}
Menghitung median data kelompok dapat menggunakan rumus di bawah ini :

\end{eulercomment}
\begin{eulerformula}
\[
M_e = Tb + p  \frac{\frac{1}2 n - F}f
\]
\end{eulerformula}
\begin{eulercomment}
Keterangan:\\
Tb = tepi bawah kelas median, ialah kelas dimana median terletak\\
p = panjang kelas median\\
n = ukuran sampel / banyak data\\
F = jumlah semua frekuensi dengan tanda kelas lebih kecil dari tanda
kelas median.\\
f = frekuensi kelas median

Untuk menghitung median data berkelompok di EMT, dapat dilakukan
dengan cara berikut:\\
1. Menentukan tepi bawah kelas (Tb), panjang kelas (P), dan tepi atas
kelas (Ta) dengan rumus :

\end{eulercomment}
\begin{eulerformula}
\[
T_b=a-0,5
\]
\end{eulerformula}
\begin{eulerformula}
\[
P=(b-a)+1
\]
\end{eulerformula}
\begin{eulerformula}
\[
T_a=b+0.5
\]
\end{eulerformula}
\begin{eulercomment}
2. Mendeskripsikan data dalam bentuk tabel, dengan perintah

\textgreater{} r=tepi bawah terkecil:panjang kelas:tepi atas terbesar;
f=[frekuensi];\\
\textgreater{} T:=r[1:jumlah kelas]' \textbar{} r[2:jumlah kelas + 1]' \textbar{} f';
writetable(T,labc=["tepi bawah","tepi atas","frekuensi"]))

3. Mendeskripsikan batas bawah kelas median, panjang kelas median,
banyak data, jumlah frekuensi sebelum kelas median, frekuensi median

\textgreater{} Tb=(tepi bawah kelas median), p=(panjang kelas median), n=(banyak
data), F=(jumlah frekuensi sebelum kelas median), f=(frekuensi kelas
median)

4. Menghitung median data dengan perintah:

\textgreater{} Tb+p*(1/2*n-F)/f

Contoh soal :\\
Berikut adalah data hasil dari pengukuran berat badan 20 siswa SD
kelas V. Dari ke 20 siswa, siswa yang mempunyai berat badan dalam
rentang 21-26 kg sebanyak 5 orang, yang mempunyai berat badan dalam
rentang 27-32 kg sebanyak 4 orang, yang mempunyai berat badan dalam
rentang 33-38 kg sebanyak 3 orang, yang mempunyai berat badan dalam
rentang 39-44 kg sebanyak 2 orang, yang mempunyai berat badan dalam
rentang 45-50 kg sebanyak 3 orang, dan yang mempunyai berat badan
51-56 kg sebanyak 3 orang. Tentukan median dari\\
data hasil pengukuran berat badan 20 siswa di SD tersebut!\\
Penyelesaian:\\
Menentukan tepi bawah kelas yang terkecil
\end{eulercomment}
\begin{eulerprompt}
>21-0.5 // menentukan tepi bawah kelas terkecil
\end{eulerprompt}
\begin{euleroutput}
  20.5
\end{euleroutput}
\begin{eulerprompt}
> (26-21)+1 // menentukan panjang kelas
\end{eulerprompt}
\begin{euleroutput}
  6
\end{euleroutput}
\begin{eulerprompt}
> 56+0.5 // tepi atas kelas terbesar
\end{eulerprompt}
\begin{euleroutput}
  56.5
\end{euleroutput}
\begin{eulerprompt}
>r=20.5:6:56.5; f=[5, 4, 3, 2, 3, 3];
>T :=r[1:6]' | r[2:7]' | f' ; writetable(T, labc=["Tb", "Ta", "frekuensi"])
\end{eulerprompt}
\begin{euleroutput}
          Tb        Ta frekuensi
        20.5      26.5         5
        26.5      32.5         4
        32.5      38.5         3
        38.5      44.5         2
        44.5      50.5         3
        50.5      56.5         3
\end{euleroutput}
\begin{eulerprompt}
>Tb=32.5, p=6, n=20, F=9, f=3
\end{eulerprompt}
\begin{euleroutput}
  32.5
  6
  20
  9
  3
\end{euleroutput}
\begin{eulerprompt}
>Tb+p*(1/2*n-F)/f
\end{eulerprompt}
\begin{euleroutput}
  34.5
\end{euleroutput}
\begin{eulercomment}
Jadi, median dari data hasil pengukuran berat badan 20 siswa SD kelas
V adalah 34.5

\end{eulercomment}
\eulersubheading{3. Mencari Modus}
\begin{eulercomment}
Modus adalah area fokus dalam analisis statistika deskriptif yang
termasuk dalam ukuran pusat data. Ini adalah nilai yang paling sering
muncul dalam kumpulan data atau nilai yang memiliki frekuensi
tertinggi dalam distribusi data. Modus dapat dibagi menjadi dua jenis,
yaitu modus untuk data tunggal dan modus untuk data kelompok.

\end{eulercomment}
\eulersubheading{a.Modus untuk data tunggal:}
\begin{eulercomment}
Menentukan modus untuk data tunggal cukup sederhana. Pertama, data
diurutkan dari nilai terkecil ke terbesar sehingga data dengan nilai
yang sama berdekatan satu sama lain. Selanjutnya, frekuensi
masing-masing data dihitung, dan data yang memiliki frekuensi
tertinggi dipilih sebagai modus.

\end{eulercomment}
\eulersubheading{b.Modus untuk data kelompok}
\begin{eulercomment}
Berikut rumus untuk mencari modus data kelompok :\\
\end{eulercomment}
\begin{eulerformula}
\[
M_o = Tb+ \frac{d_1}{d_1+d_2} c
\]
\end{eulerformula}
\begin{eulercomment}
Keterangan :\\
Tb = Tepi bawah\\
d1 = selisih f modus dengan f sebelumnya\\
d2 = selisih f modus dengan f sesudahnya\\
\end{eulercomment}
\begin{eulerttcomment}
 c = Panjang kelas
\end{eulerttcomment}
\begin{eulercomment}

Untuk menghitung modus data berkelompok di EMT, dapat dilakukan dengan
cara berikut:\\
1. Menentukan tepi bawah kelas (Tb), panjang kelas (P), dan tepi atas
kelas (Ta) dengan rumus :\\
\end{eulercomment}
\begin{eulerformula}
\[
T_b=a-0,5
\]
\end{eulerformula}
\begin{eulerformula}
\[
P=(b-a)+1
\]
\end{eulerformula}
\begin{eulerformula}
\[
T_a=b+0.5
\]
\end{eulerformula}
\begin{eulercomment}
dimana a = batas bawah kelas dan b = batas atas kelas

2. Mendeskripsikan data dalam bentuk tabel, dengan perintah\\
\textgreater{} r=tepi bawah terkecil:panjang kelas:tepi atas terbesar;
v=[frekuensi];\\
\textgreater{} T:=r[1:jumlah kelas]’ \textbar{} r[2:jumlah kelas + 1]’ \textbar{} f’;
writetable(T,labc=[”tepi bawah”,”tepi atas”,”frekuensi”]))

3. Mendeskripsikan tepi bawah kelas modus, panjang kelas modus,selisih
frekuensi modus dengan\\
frekuensi sebelumnya, selisih frekuensi modus dengan frekuensi
sesudahnya\\
\textgreater{} Tb=(tepi bawah kelas modus), p=(panjang kelas modus), d1=(selisih
frekuensi modus dengan frekuensi sebelumnya), d2=(selisih frekuensi
dengan frekuensi sesudahnya)

4. Menghitung modus dengan perintah:\\
\textgreater{} Tb+p*d1/(d1+d2)

Contoh soal\\
Diketahui sebuah data kelompok sebagai berikut :
\end{eulercomment}
\begin{eulerprompt}
>printfile("Tabel modus data kelompok.dat",8) 
\end{eulerprompt}
\begin{euleroutput}
        Kelas     Frekuensi
        20-29         3
        30-39         7
        40-49         8
        50-59         12
        60-69         9
        70-79         6
        80-89         5
\end{euleroutput}
\begin{eulercomment}
Berapakah modus dari data tersebut?
\end{eulercomment}
\begin{eulerprompt}
>20-0.5  //menentukan tepi bawah kelas
\end{eulerprompt}
\begin{euleroutput}
  19.5
\end{euleroutput}
\begin{eulerprompt}
>(29-20)+1 //menentukan panjang kelas
\end{eulerprompt}
\begin{euleroutput}
  10
\end{euleroutput}
\begin{eulerprompt}
>89+0.5 //menentukan tepi atas
\end{eulerprompt}
\begin{euleroutput}
  89.5
\end{euleroutput}
\begin{eulerprompt}
>r=19.5:10:89.5; f=[3, 7, 8, 12, 9, 6, 5];
>T:=r[1:7]' | r[2:8]' | f'; writetable(T,labc=["Tb", "Ta", "frekuensi"])
\end{eulerprompt}
\begin{euleroutput}
          Tb        Ta frekuensi
        19.5      29.5         3
        29.5      39.5         7
        39.5      49.5         8
        49.5      59.5        12
        59.5      69.5         9
        69.5      79.5         6
        79.5      89.5         5
\end{euleroutput}
\begin{eulercomment}
Berdasarkan tabel di atas, modus berada pada kelas 49.5-59.5
\end{eulercomment}
\begin{eulerprompt}
>Tb=49.5, p=10, d1=12-8, d2=12-9
\end{eulerprompt}
\begin{euleroutput}
  49.5
  10
  4
  3
\end{euleroutput}
\begin{eulerprompt}
>Tb+p*d1/(d1+d2)
\end{eulerprompt}
\begin{euleroutput}
  55.2142857143
\end{euleroutput}
\begin{eulercomment}
Jadi, modus dari data kelompok di atas adalah 55.2142857143
\end{eulercomment}
\eulersubheading{4. Mencari varians/ragam}
\begin{eulercomment}
Varians digunakan untuk mengetahui bagaimana sebaran data terhadap
mean atau nilai rata-rata. Sederhananya, varians adalah ukuran
statistik jauh dekatnya penyebaran data dari nilai rata-ratanya. Dalam
mencari ragam dapat dikelompokkan menjadi 2 jenis yaitu sebagai
berikut :

\end{eulercomment}
\eulersubheading{a.Varians data tunggal}
\begin{eulercomment}
Rumus untuk varians data tunggal berikut :\\
1) Untuk populasi

\end{eulercomment}
\begin{eulerformula}
\[
{\sigma}^2=\frac{\sum_{i=1}^{n} (x-\mu)^2}n
\]
\end{eulerformula}
\begin{eulercomment}
2) Untuk sampel

\end{eulercomment}
\begin{eulerformula}
\[
S^2=\frac{\sum_{i=1}^{n} (x-\bar{x})^2}{n-1}
\]
\end{eulerformula}
\begin{eulercomment}
Pada EMT, untuk menemukan suatu Ragam data tunggal dapat menggunakan
perintah berikut:

\textgreater{} mean(dev\textasciicircum{}2)

Contoh soal\\
Hitunglah nilai varians dari data sampel nilai siswa: 9, 10, 6, 7!
\end{eulercomment}
\begin{eulerprompt}
>data=[9, 10, 6, 7]; //mendefinisikan data
>urut=sort(data) //mengurutkan data
\end{eulerprompt}
\begin{euleroutput}
  [6,  7,  9,  10]
\end{euleroutput}
\begin{eulerprompt}
>xbar=mean(urut) //menghitung rata rata dari data
\end{eulerprompt}
\begin{euleroutput}
  8
\end{euleroutput}
\begin{eulerprompt}
>dev= urut-xbar 
\end{eulerprompt}
\begin{euleroutput}
  [-2,  -1,  1,  2]
\end{euleroutput}
\begin{eulerprompt}
>varians=mean(dev^2) //menghitung varians
\end{eulerprompt}
\begin{euleroutput}
  2.5
\end{euleroutput}
\begin{eulercomment}
Jadi, varians dari data sampel tersebut adalah 2.5

\end{eulercomment}
\eulersubheading{b. Varians data kelompok}
\begin{eulercomment}
Rumus untuk varians data kelompok sebagai berikut :\\
1) Untuk populasi

\end{eulercomment}
\begin{eulerformula}
\[
{\sigma}^2=\frac{\sum_{i=1}^{n} f_i (x_i-\mu)^2}{\sum_{i=1}^{n}f_i}
\]
\end{eulerformula}
\begin{eulercomment}
2) Untuk sampel

\end{eulercomment}
\begin{eulerformula}
\[
S^2=\frac{\sum_{i=1}^{n} f_i (x_i-\bar{x})^2}{\sum_{i=1}^{n}f_i-1}
\]
\end{eulerformula}
\begin{eulercomment}
Pada EMT, untuk menemukan Ragam data berkelompk dapat menggunakan
perintah berikut:

1. Menentukan tepi bawah kelas (Tb), panjang kelas (P), dan tepi atas
kelas (Ta) dengan rumus :

\end{eulercomment}
\begin{eulerformula}
\[
T_b=a-0,5
\]
\end{eulerformula}
\begin{eulerformula}
\[
P=(b-a)+1
\]
\end{eulerformula}
\begin{eulerformula}
\[
T_a=b+0.5
\]
\end{eulerformula}
\begin{eulercomment}
dengan a = batas bawah kelas dan b = batas atas kelas

2. Mendeskripsikan data dalam bentuk tabel, dengan perintah

\textgreater{} r=tepi bawah terkecil:panjang kelas:tepi atas terbesar;
f=[frekuensi];\\
\textgreater{} T:=r[1:jumlah kelas]' \textbar{} r[2:jumlah kelas + 1]' \textbar{} f';
writetable(T,labc=["tepi bawah","tepi atas","frekuensi"])

3. Menghitung Ragam dengan perintah

\textgreater{} (T[,1]+T[,2])/2; t=fold(r,[0.5,0.5]);m=mean(t,f);\\
\textgreater{} sum(f*(t-m)\textasciicircum{}2)/sum(f)  //untuk populasi\\
\textgreater{} sum(f*(t-m)\textasciicircum{}2)/(sum(f)-1)  //untuk sampel

Contoh soal\\
Tentukan varians data sampel dari tabel berikut :
\end{eulercomment}
\begin{eulerprompt}
>printfile("Tabel data kelompok varians.dat",7) 
\end{eulerprompt}
\begin{euleroutput}
     Nilai          Frekuensi
     63-67              3
     68-72              2
     73-77              7
     78-82              3
     83-87              4
     88-92              1
\end{euleroutput}
\begin{eulerprompt}
>63-0.5  //tapi bawah terkecil
\end{eulerprompt}
\begin{euleroutput}
  62.5
\end{euleroutput}
\begin{eulerprompt}
>(67-63)+1  //panjang kelas
\end{eulerprompt}
\begin{euleroutput}
  5
\end{euleroutput}
\begin{eulerprompt}
>92+0.5  //tepi atas terbesar
\end{eulerprompt}
\begin{euleroutput}
  92.5
\end{euleroutput}
\begin{eulerprompt}
>r=62.5:5:92.5; f=[3,2,7,3,4,1];
>T:=r[1:6]' | r[2:7]' | f'; writetable(T, labc=["tepi bawah", "tepi atas", "frekuensi"])
\end{eulerprompt}
\begin{euleroutput}
   tepi bawah tepi atas frekuensi
         62.5      67.5         3
         67.5      72.5         2
         72.5      77.5         7
         77.5      82.5         3
         82.5      87.5         4
         87.5      92.5         1
\end{euleroutput}
\begin{eulerprompt}
>(T[,1]+T[,2])/2; t=fold(r,[0.5,0.5]) 
\end{eulerprompt}
\begin{euleroutput}
  [65,  70,  75,  80,  85,  90]
\end{euleroutput}
\begin{eulerprompt}
>m=mean(t,f)
\end{eulerprompt}
\begin{euleroutput}
  76.5
\end{euleroutput}
\begin{eulerprompt}
>sum(f*(t-m)^2)/(sum(f)-1)
\end{eulerprompt}
\begin{euleroutput}
  52.8947368421
\end{euleroutput}
\begin{eulercomment}
Jadi, varians dari data kelompok dari tabel di atas adalah
52.8947368421
\end{eulercomment}
\begin{eulerprompt}
> 
\end{eulerprompt}
\eulersubheading{5. Mencari Simpangan Baku}
\begin{eulercomment}
Standar Deviasi atau simpangan baku adalah akar dari ragam/varians.
Untuk nenetukan nilai standar deviasi, caranya:

\end{eulercomment}
\begin{eulerformula}
\[
\sigma=\sqrt{\sigma^2}
\]
\end{eulerformula}
\begin{eulerformula}
\[
atau
\]
\end{eulerformula}
\begin{eulerformula}
\[
S=\sqrt{S^2}
\]
\end{eulerformula}
\begin{eulercomment}
\end{eulercomment}
\eulersubheading{a. Simpangan baku data tunggal}
\begin{eulercomment}
Untuk data tunggal, simpangan baku populasi atau sampel dapat
dirumuskan sebagai berikut:

1) Untuk populasi

\end{eulercomment}
\begin{eulerformula}
\[
{\sigma}=\sqrt{\frac{\sum_{i=1}^{n} (x-\mu)^2}n}
\]
\end{eulerformula}
\begin{eulercomment}
2) Untuk sampel

\end{eulercomment}
\begin{eulerformula}
\[
S=\sqrt{\frac{\sum_{i=1}^{n} (x-\bar{x})^2}{n-1}}
\]
\end{eulerformula}
\begin{eulercomment}
Pada EMT, untuk menemukan suatu Ragam data tunggal dapat menggunakan
perintah berikut:

\textgreater{} sqrt(mean(dev\textasciicircum{}2))

Contoh soal :\\
1. Simpangan baku untuk data 70,80,40,25,65,87,97,59,24,77,45 adalah\\
Jawab :
\end{eulercomment}
\begin{eulerprompt}
>data=[70,80,40,25,65,87,97,59,24,77,45];
>urut=sort(data)
\end{eulerprompt}
\begin{euleroutput}
  [24,  25,  40,  45,  59,  65,  70,  77,  80,  87,  97]
\end{euleroutput}
\begin{eulerprompt}
>x=mean(urut)
\end{eulerprompt}
\begin{euleroutput}
  60.8181818182
\end{euleroutput}
\begin{eulerprompt}
>dev=urut-x
\end{eulerprompt}
\begin{euleroutput}
  [-36.8182,  -35.8182,  -20.8182,  -15.8182,  -1.81818,  4.18182,
  9.18182,  16.1818,  19.1818,  26.1818,  36.1818]
\end{euleroutput}
\begin{eulerprompt}
>varians=mean(dev^2)
\end{eulerprompt}
\begin{euleroutput}
  550.148760331
\end{euleroutput}
\begin{eulerprompt}
>simpanganbaku= sqrt(varians)
\end{eulerprompt}
\begin{euleroutput}
  23.4552501656
\end{euleroutput}
\begin{eulercomment}
Jadi, simpang baku data tersebut adalah 23.4552501656


\end{eulercomment}
\eulersubheading{b. Simpangan baku data kelompok}
\begin{eulerttcomment}
 Untuk data berkelompok dapat dirumuskan seperti berikut:
\end{eulerttcomment}
\begin{eulercomment}
1) Untuk populasi

\end{eulercomment}
\begin{eulerformula}
\[
{\sigma}=\sqrt{\frac{\sum_{i=1}^{n} f_i (x_i-\mu)^2}{\sum_{i=1}^{n}f_i}}
\]
\end{eulerformula}
\begin{eulercomment}
2) Untuk sampel

\end{eulercomment}
\begin{eulerformula}
\[
S=\sqrt{\frac{\sum_{i=1}^{n} f_i (x_i-\bar{x})^2}{\sum_{i=1}^{n}f_i-1}}
\]
\end{eulerformula}
\begin{eulercomment}
Pada EMT, untuk menemukan Ragam data berkelompk dapat menggunakan
perintah berikut:

1. Menentukan tepi bawah kelas (Tb), panjang kelas (P), dan tepi atas
kelas (Ta) dengan rumus :

\end{eulercomment}
\begin{eulerformula}
\[
T_b=a-0,5
\]
\end{eulerformula}
\begin{eulerformula}
\[
P=(b-a)+1
\]
\end{eulerformula}
\begin{eulerformula}
\[
T_a=b+0.5
\]
\end{eulerformula}
\begin{eulercomment}
dengan a = batas bawah kelas dan b = batas atas kelas

2. Mendeskripsikan data dalam bentuk tabel, dengan perintah

\textgreater{} r=tepi bawah terkecil:panjang kelas:tepi atas terbesar;
f=[frekuensi];\\
\textgreater{} T:=r[1:jumlah kelas]' \textbar{} r[2:jumlah kelas + 1]' \textbar{} f';
writetable(T,labc=["tepi bawah","tepi atas","frekuensi"])

3. Menghitung Ragam dengan perintah

\textgreater{} (T[,1]+T[,2])/2; t=fold(r,[0.5,0.5]);m=mean(t,f);\\
\textgreater{} sqrt(sum(f*(t-m)\textasciicircum{}2)/sum(f))     // untuk populasi\\
\textgreater{} sqrt(sum(f*(t-m)\textasciicircum{}2)/(sum(f)-1))     // untuk sampel

Contoh soal :\\
Simpangan baku dari tabel dibawah ini adalah
\end{eulercomment}
\begin{eulerprompt}
>printfile("Tabel simpangan baku data kelompok.dat",7)
\end{eulerprompt}
\begin{euleroutput}
        Interval nilai     Frekuensi
            41-45              10
            46-50              12
            51-55              18
            56-60              34
            61-65              20
            66-70              6
\end{euleroutput}
\begin{eulerprompt}
>41-0.5 //tepi bawah terkecil
\end{eulerprompt}
\begin{euleroutput}
  40.5
\end{euleroutput}
\begin{eulerprompt}
>(45-41)+1 //panjang kelas
\end{eulerprompt}
\begin{euleroutput}
  5
\end{euleroutput}
\begin{eulerprompt}
>70+0.5 //tepi atas terbesar
\end{eulerprompt}
\begin{euleroutput}
  70.5
\end{euleroutput}
\begin{eulerprompt}
>r=40.5:5:70.5; f=[10,12,18,34,20,6];
>T:=r[1:6]' | r[2:7]' | f'; writetable(T,labc=["tepi bawah", "tepi atas", "frekuensi"])
\end{eulerprompt}
\begin{euleroutput}
   tepi bawah tepi atas frekuensi
         40.5      45.5        10
         45.5      50.5        12
         50.5      55.5        18
         55.5      60.5        34
         60.5      65.5        20
         65.5      70.5         6
\end{euleroutput}
\begin{eulerprompt}
>(T[,1]+T[,2])/2; t=fold(r,[0.5,0.5]); m=mean(t,f);
\end{eulerprompt}
\begin{eulercomment}
karena data tersebut merupakan data sampel, maka menggunakan rumus
berikut
\end{eulercomment}
\begin{eulerprompt}
>sqrt(sum(f*(t-m)^2)/(sum(f)-1))
\end{eulerprompt}
\begin{euleroutput}
  6.81649810861
\end{euleroutput}
\begin{eulercomment}
Jadi, simpangan baku data kelompok tersebut adalah 6.81649810861

\end{eulercomment}
\eulersubheading{6.Mencari Jangkauan/Range}
\begin{eulercomment}
Jangkauan, atau biasa disebut range, merupakan perbedaan antara nilai
data tertinggi dan nilai data terendah dalam suatu set data. Metode
pencarian jangkauan berbeda antara data tunggal dan data kelompok.

\end{eulercomment}
\eulersubheading{a. Jangkauan/Range Data Tunggal}
\begin{eulercomment}
Bila ada sekumpulan data tunggal terurut dari yang terkecil sampai
terbesar adalah

\end{eulercomment}
\begin{eulerformula}
\[
x_1, x_2,..., x_n
\]
\end{eulerformula}
\begin{eulercomment}
maka jangkauannya adalah:

\end{eulercomment}
\begin{eulerformula}
\[
Jangkauan = x_n-n_1
\]
\end{eulerformula}
\begin{eulercomment}
Untuk menemukan jangkauan data tunggal di EMT dapat menggunakan
perintah berikut:

\textgreater{} x=[data]; max(x)-min(x)

Contoh soal

Jangkauan dari data 30,60,87,55,87,98,22,75,81,70,69,84,75 adalah...
\end{eulercomment}
\begin{eulerprompt}
>x=[30,60,87,55,87,98,22,75,81,70,69,84,75]; max(x)- min(x)
\end{eulerprompt}
\begin{euleroutput}
  76
\end{euleroutput}
\begin{eulercomment}
Jadi, jangkauan dari data tersebut adalah 76

\end{eulercomment}
\eulersubheading{b. Jangkauan data kelompok}
\begin{eulercomment}
Jangkauan pada data berkelompok adalah selisih antara batas atas dari
kelas tertinggi dengan batas bawah dari kelas terendah.

Pada EMT,  untuk menemukan jangkauan dari data berkelompok dapat
menggunakan perintah berikut:

1. Menentukan tepi bawah kelas (Tb), panjang kelas (P), dan tepi atas
kelas (Ta) dengan rumus :

\end{eulercomment}
\begin{eulerformula}
\[
T_b=a-0,5
\]
\end{eulerformula}
\begin{eulerformula}
\[
P=(b-a)+1
\]
\end{eulerformula}
\begin{eulerformula}
\[
T_a=b+0.5
\]
\end{eulerformula}
\begin{eulercomment}
dengan a = batas bawah kelas dan b = batas atas kelas

2. Mendeskripsikan data dalam bentuk tabel, dengan perintah

\textgreater{} r=tepi bawah terkecil:panjang kelas:tepi atas terbesar;
f=[frekuensi];\\
\textgreater{} T:=r[1:jumlah kelas]' \textbar{} r[2:jumlah kelas + 1]' \textbar{} f';
writetable(T,labc=["tepi bawah","tepi atas","frekuensi"])

3. Menghitung jangkauan data berkelompok

\textgreater{} max(transpose(T[,2]))-min(transpose(T[,1]))\\
Contoh soal :\\
Berikut adalah data hasil dari pengukuran berat badan 20 siswa SD
kelas V. Dari ke 20 siswa,siswa yang mempunyai berat badan dalam
rentang 21-26 kg sebanyak 5 orang, yang mempunyai berat badan dalam
rentang 27-32 kg sebanyak 4 orang, yang mempunyai berat badan dalam
rentang 33-38 kg sebanyak 3 orang, yang mempunyai berat badan dalam
rentang 39-44 kg sebanyak 2 orang, yang mempunyai berat badan dalam
rentang 45-50 kg sebanyak 3 orang, dan yang mempunyai berat badan
51-56 kg sebanyak 3 orang. Tentukan jangkauan dari\\
data hasil pengukuran berat badan 20 siswa di SD tersebut!\\
Jawab :
\end{eulercomment}
\begin{eulerprompt}
>printfile("Tabel jangkauan data kelompok.dat",7) //menyederhanakan informasi
\end{eulerprompt}
\begin{euleroutput}
        Interval nilai     Frekuensi
            21-26              5
            27-32              4
            33-38              3
            39-44              2
            45-50              3
            51-56              3
\end{euleroutput}
\begin{eulerprompt}
>21-0.5 //tepi bawah terkecil
\end{eulerprompt}
\begin{euleroutput}
  20.5
\end{euleroutput}
\begin{eulerprompt}
>(26-21)+1 //panjang kelas
\end{eulerprompt}
\begin{euleroutput}
  6
\end{euleroutput}
\begin{eulerprompt}
>56+0.5 //tepi atas terbesar
\end{eulerprompt}
\begin{euleroutput}
  56.5
\end{euleroutput}
\begin{eulerprompt}
>r=20.5:6:56.5; f=[5,4,3,2,3,3];
>T:=r[1:6]' | r[2:7]' | f'; writetable(T,labc=["tepi bawah","tepi atas","frekuensi"])
\end{eulerprompt}
\begin{euleroutput}
   tepi bawah tepi atas frekuensi
         20.5      26.5         5
         26.5      32.5         4
         32.5      38.5         3
         38.5      44.5         2
         44.5      50.5         3
         50.5      56.5         3
\end{euleroutput}
\begin{eulerprompt}
>max(transpose(T[,2]))-min(transpose(T[,1]))
\end{eulerprompt}
\begin{euleroutput}
  36
\end{euleroutput}
\begin{eulercomment}
Jadi, Jangkauan dari data kelompok tersebut adalah 36
\end{eulercomment}
\eulersubheading{7.Menentukan ukuran letak}
\begin{eulercomment}
Ukuran letak merupakan ukuran untuk melihat dimana letak salah satu
data dari sekumpulan banyak data yang ada. Yang termasuk ukuran ukuran
letak antara lain adalah kuartil(Q), desil(D) dan persentil(P). Dalam
menentukan ke-3 nya yang harus diingat adalah mengurutkan distribusi
data dari yang terkecil sampai terbesar

1. Kuartil\\
Dalam EMT untuk menghitung kuartil bisa dilakukan dengan perintah\\
\textgreater{}quartiles(data)\\
perintah tersebut akan menghasilkan nilai Q1, Q2, Q3, nilai minimum
dan nilai maksimum dari suatu data\\
2. Desil\\
Dalam EMT untuk menghitung desil bisa dilakukan dengan perintah\\
\textgreater{}quantile(data)\\
3. Persentil\\
Dalam EMT untuk menghitung persentil bisa dilakukan dengan perintah\\
\textgreater{}quantile(data)\\
perintah "\textgreater{}quantile(data)" dapat digunakan untuk menentukan desil dan
persentil perbedaannya tergantung pada nilai dari pembaginya

Contoh soal\\
1. Tentukan Q1,Q2 dan Q3 dari data :
7,3,8,5,9,4,8,3,10,2,7,6,8,7,2,6,9.
\end{eulercomment}
\begin{eulerprompt}
>data=[7,3,8,5,9,4,8,3,10,2,7,6,8,7,2,6,9];
>urut=sort(data)
\end{eulerprompt}
\begin{euleroutput}
  [2,  2,  3,  3,  4,  5,  6,  6,  7,  7,  7,  8,  8,  8,  9,  9,  10]
\end{euleroutput}
\begin{eulerprompt}
>quartiles(urut)
\end{eulerprompt}
\begin{euleroutput}
  [2,  3.5,  7,  8,  10]
\end{euleroutput}
\begin{eulercomment}
dari hasil di atad diperoleh nilai sebagai berikut :\\
Nilai minimal data = 2\\
Q1=3.5\\
Q2=7\\
Q3=8\\
Nilai maksimal data = 10

2. Tentukan D8 dari data : 6,3,8,9,5,9,9,7,5,7,4,5,8,3,7,6
\end{eulercomment}
\begin{eulerprompt}
>data=[6,3,8,9,5,9,9,7,5,7,4,5,8,3,7,6];
>urut=sort(data)
\end{eulerprompt}
\begin{euleroutput}
  [3,  3,  4,  5,  5,  5,  6,  6,  7,  7,  7,  8,  8,  9,  9,  9]
\end{euleroutput}
\begin{eulerprompt}
>quantile(urut,0.8)  //nilai 0.8 diapatkan karena kita akan mencari D8
\end{eulerprompt}
\begin{euleroutput}
  8
\end{euleroutput}
\begin{eulercomment}
Jadi, nilai dari D8 berdasarkan perhitungan di atas adalah 8

3.Tentukan persentil ke-65 dari data : 6,5,8,7,9,4,5,8,4,7,8,5,8,4,5
\end{eulercomment}
\begin{eulerprompt}
>data=[6,5,8,7,9,4,5,8,4,7,8,5,8,4,5];
>urut=sort(data)
\end{eulerprompt}
\begin{euleroutput}
  [4,  4,  4,  5,  5,  5,  5,  6,  7,  7,  8,  8,  8,  8,  9]
\end{euleroutput}
\begin{eulerprompt}
>quantile(urut,65%)
\end{eulerprompt}
\begin{euleroutput}
  7.1
\end{euleroutput}
\begin{eulercomment}
\begin{eulercomment}
\eulerheading{Sub Topik 6: Menggambar Grafik  Statistika}
\begin{eulercomment}
\end{eulercomment}
\eulersubheading{Diagram Kotak}
\begin{eulercomment}
Diagram kotak atau box plot merupakan ringkasan distribusi sampel yang
disajikan secara grafis yang bisa menggambarkan bentuk distribusi data
(skewness), ukuran tendensi sentral dan ukuran penyebaran (keragaman)
data pengamatan. Diagram kotak sering digunakan ketika  jumlah
distribusi data perlu dibandingkan. Diagram kotak menyajikan informasi
tentang nilai--nilai inti dalam distribusi data termasuk juga
pencilan. Pencilan adalah titik data yang terpaut jauh dari titik data
lainnya.

Contoh:\\
Diketahui data berat badan mahasiswa di Universitas A sebagai berikut.
\end{eulercomment}
\begin{eulerprompt}
>A=[55,50,33,42,44,37,63,74,56,34,51,43,45,39,64,77,60,35,53,43,48,41,65,87,61,36,54,44,49,41,66,89]
\end{eulerprompt}
\begin{euleroutput}
  [55,  50,  33,  42,  44,  37,  63,  74,  56,  34,  51,  43,  45,  39,
  64,  77,  60,  35,  53,  43,  48,  41,  65,  87,  61,  36,  54,  44,
  49,  41,  66,  89]
\end{euleroutput}
\begin{eulercomment}
Buatlah diagram kotak (box plot) kemudian tuliskan interpretasinya.
\end{eulercomment}
\begin{eulerprompt}
>boxplot(A):
\end{eulerprompt}
\eulerimg{25}{images/Kelompok 6_EMT4Statistika-010.png}
\begin{eulercomment}
Dari gambar box plot berat  badan mahasiswa Universitas  A, sepintas
kita bisa menentukan beberapa ukuran statistik, meskipun tidak persis
sekali. Nilai statistik pada badan boxplot berkisar pada: Nilai
Minimum = 33 , Q1 = 41.5 , Median (Q2) = 49.5 , Q3 = 62 , Nilai
Maksimum  = 89 . Sebaran data tidak simetris, melainkan menjulur ke
arah kanan (postively skewness). Karena nilai jarak Q1 dengan Q2 lebih
pendek dari jarak Q2 dengan Q3, maka data lebih terpusat di kiri. Akan
tetapi data tersebut tergolong cenderung mesokurtik karena jarak IQR
dengan panjang hampir sama, dengan data berpusat di angka 49.5


Adapun contoh perbandingan 10 simulasi 500 nilai terdistribusi normal
menggunakan box plot dan terdapat pencilan sebagai berikut.

\end{eulercomment}
\begin{eulerprompt}
> p=normal(10,500); boxplot(p):
\end{eulerprompt}
\eulerimg{25}{images/Kelompok 6_EMT4Statistika-011.png}
\begin{eulercomment}
pada diagram diatas, adalah membuat boxplot distribusi normal dengan
rata-rata 10 dan standar deviasi 500. Boxplot adalah representasi
grafis dari lokalitas, penyebaran, dan kecondongan sekelompok data
numerik melalui kuartil mereka\\
2

\end{eulercomment}
\eulersubheading{Diagram Batang}
\begin{eulercomment}
Diagram batang adalah representasi visual dari data yang menggunakan
balok atau kolom vertikal untuk mewakili kategori, nilai atau variabel
tertentu. Setiap kolom yang ada pada diagram  batang memiliki
frekuensi atau jumlah dalam kategori tersebut.

Contoh:

Kita akan membuat diagram batang secara random.
\end{eulercomment}
\begin{eulerprompt}
>columnsplot(cumsum(random(6)),style="/",color=red):
\end{eulerprompt}
\eulerimg{25}{images/Kelompok 6_EMT4Statistika-012.png}
\begin{eulerprompt}
>columnsplot(cumsum(random(15)),style="-",color=black):
\end{eulerprompt}
\eulerimg{25}{images/Kelompok 6_EMT4Statistika-013.png}
\begin{eulerprompt}
>columnsplot(cumsum(random(3)),style="|",color=orange):
\end{eulerprompt}
\eulerimg{25}{images/Kelompok 6_EMT4Statistika-014.png}
\begin{eulercomment}
Selanjutnya kita akan mencoba  membuat diagram batang penjualan yang
menggunakan variabel.
\end{eulercomment}
\begin{eulerprompt}
>months=["Januari","Februari","Maret","April","Mei"];
>values=[20,50,40,70,30];
>columnsplot(values,lab=months,color=yellow);
>title("Data Penjualan Beras Toko Kuning pada tahun 2023"):
\end{eulerprompt}
\eulerimg{25}{images/Kelompok 6_EMT4Statistika-015.png}
\begin{eulercomment}
Perintah "columnsplot(values,lab=months,color=yellow);" merupakan
sintaks untuk membuat diagram batang dengan menggunakan nilai dari
variabel "values", label bulan dari variabel "months", dan warna
kuning

Dari diagram batang tersebut kita bisa mengetahui data penjualan toko
kuning selama lima bulan pada tahun 2023  yaitu, pada bulan Januari,
Februari, Maret , April, Mei. Januari terjual 20 ton beras, Februari
terjual 50 ton beras, Maret terjual 40 ton beras, April terjual 70 ton
beras, dan Mei terjual 30 ton beras.

\end{eulercomment}
\eulersubheading{Diagram Lingkaran}
\begin{eulercomment}
Diagram lingkaran merupakan penyajian statistik data tunggal dalam\\
bentuk lingkaran yang dibagi menjadi beberapa juring atau sektor yang\\
menggambarkan banyak frekuensi untuk setiap data.Diagram lingkaran\\
tidak menampilkan informasi frekuensi dari masing-masing data secara\\
detail.
\end{eulercomment}
\begin{eulerprompt}
>CP:=[rgb(0.5,0.5,0.5),red,yellow,green,rgb(0.9,0,0)]
\end{eulerprompt}
\begin{euleroutput}
  [5.87532e+07,  2,  15,  3,  6.54049e+07]
\end{euleroutput}
\begin{eulerprompt}
>i=[1,2,3,4,5]; piechart(values[i],color=CP[i],lab=months[i]):
\end{eulerprompt}
\eulerimg{25}{images/Kelompok 6_EMT4Statistika-016.png}
\begin{eulercomment}
RGB adalah singkatan dari Red, Green, and Blue, dan setiap parameter
mendefinisikan intensitas warna dengan nilai antara 0 dan 1. Warna
pertama dalam daftar adalah warna abu-abu dengan jumlah merah, hijau,
dan biru yang sama. Warna kedua merah, ketiga kuning, dan keempat
hijau. Warna terakhir adalah warna merah dengan lebih banyak merah
daripada hijau atau biru.

\end{eulercomment}
\eulersubheading{Diagram Bintang}
\begin{eulercomment}
Diagram bintang, terkadang disebut diagram radar atau diagram web,
adalah metode perangkat grafis yang digunakan untuk menampilkan data
multivariat. Multivariat dalam pengertian ini mengacu pada memiliki
banyak karakteristik untuk diamati. Variabelnya juga harus berupa
nilai yang berkisar.\\
Diagram bintang terdiri dari rangkaian jari-jari bersudut sama, yang
disebut jari-jari, dengan masing-masing jari mewakili salah satu
variabel. Panjang jari-jari data sebanding dengan besaran variabel
pada titik data relatif terhadap besaran maksimum variabel di
seluruh titik data.
\end{eulercomment}
\begin{eulerprompt}
>starplot(normal(1,15)+16,lab=1:15,>rays):
\end{eulerprompt}
\eulerimg{25}{images/Kelompok 6_EMT4Statistika-017.png}
\begin{eulerprompt}
>starplot(values,lab=months,>rays):
\end{eulerprompt}
\eulerimg{25}{images/Kelompok 6_EMT4Statistika-018.png}
\begin{eulercomment}
Syntax starplot(values,lab=months,rays) adalah perintah untuk membuat
grafik bintang (star plot) dengan menggunakan nilai-nilai yang
diberikan dalam vektor values, label sumbu yang diberikan dalam vektor
months, dan jumlah rays yang menentukan jumlah garis radial yang
digunakan dalam grafik


\end{eulercomment}
\eulersubheading{Diagram Impuls}
\begin{eulercomment}
Impuls (impulse) adalah perubahan momentum. Contohnya adalah sebuah
bola bermassa yang tengah ditendang, bola menggelinding yang
dihentikan, bola jatuh yang memantul, mobil yang menabrak tembok,
telur jatuh yang pecah.\\
Berikut adalah plot impuls dari data acak 1 sampai 20, terdistribusi
secara merata di [0,1].
\end{eulercomment}
\begin{eulerprompt}
>plot2d(makeimpulse(1:20,random(1,20)),>bar):
\end{eulerprompt}
\eulerimg{25}{images/Kelompok 6_EMT4Statistika-019.png}
\begin{eulercomment}
Tetapi untuk data yang terdistribusi secara eksponensial, kita mungkin
memerlukan plot logaritmik.
\end{eulercomment}
\begin{eulerprompt}
> logimpulseplot(1:20,-log(random(1,20))*10):
\end{eulerprompt}
\eulerimg{25}{images/Kelompok 6_EMT4Statistika-020.png}
\begin{eulercomment}
Jadi gambar grafiknya terlihat naik turun (mengalami perubahan).
\end{eulercomment}
\eulersubheading{Histogram}
\begin{eulercomment}
Histogram adalah representasi grafis (diagram) yang mengatur dan
menampilkan frekuensi data sampel pada rentang tertentu. Frekuensi
data yang ada pada masing-masing kelas direpresentasikan dengan bentuk
grafik diagram batang atau kolom.
\end{eulercomment}
\begin{eulerprompt}
>aspect(1); plot2d(random(100),>histogram):
\end{eulerprompt}
\eulerimg{25}{images/Kelompok 6_EMT4Statistika-021.png}
\begin{eulerprompt}
>r=150:5:185; v=[22,71,136,150,139,71,32];
>plot2d(r,v,a=150,b=185,c=0,d=150,bar=1,style="/"):
\end{eulerprompt}
\eulerimg{25}{images/Kelompok 6_EMT4Statistika-022.png}
\begin{eulercomment}
Pola "r=150:5:185" berarti bahwa nilai r dimulai dari 150, kemudian
bertambah 5 setiap kali, dan berakhir saat mencapai atau melebihi 185.
Dengan pola ini, kita dapat menentukan nilai-nilai r yang sesuai.

Dari data yang diperoleh dapat diketahui bahwa dari rentang kelas
150-155 memiliki frekuensi 22, rentang kelas 155-160 memiliki
frekuensi 71,  dan seterusnya.

\end{eulercomment}
\eulersubheading{Kurva Fungsi Kerapatan Probabilitas}
\begin{eulercomment}
Secara teoritis kurva probabilitas populasi diwakili oleh poligon
frekuensi relatif yang dimuluskan (variabel acak  kontiniu
diperlakukan seperti variabel acak diskrit yang rapat).Karena itu
fungsi dari variabel acak kontiniu merupakan fungsi kepadatan
probabilitas (probability density function – pdf). Pdf menggambarkan
besarnya probabilitas per unit interval nilai variabel acaknya.
\end{eulercomment}
\begin{eulerprompt}
>plot2d("qnormal(x,0,1)",-5,5);  ...
>plot2d("qnormal(x,0,1)",a=1,b=4,>add,>filled):
\end{eulerprompt}
\eulerimg{25}{images/Kelompok 6_EMT4Statistika-023.png}
\begin{eulercomment}
Perintah "plot2d("qnormal(x,0,1)",-5,5)" digunakan untuk membuat plot
dari distribusi normal dengan mean 0 dan standard deviation 1 di
rentang -5 hingga 5

Probabilitas variabel acak x yang terletak antara 1 dan 4 memenuhi\\
P(1\textless{}X\textless{}4)= luas daerah hijau

\end{eulercomment}
\eulersubheading{Kurva Fungsi Distribusi Kumulatif}
\begin{eulercomment}
Cumulative Distribution Function (CDF) atau fungsi distribusi
kumulatif adalah fungsi matematika yang digunakan untuk menghitung
probabilitas variabel acak diskrit atau kontinu. CDF memberikan
probabilitas bahwa variabel acak akan menghasilkan nilai kurang dari
atau sama dengan nilai tertentu. Dalam hal ini, CDF dapat digunakan
untuk menghitung probabilitas kumulatif dari variabel acak.

Berikut merupakan contoh kurva fungsi distribusi kumulatif kontinu:
\end{eulercomment}
\begin{eulerprompt}
>splot2d("normaldis",-3,5):
\end{eulerprompt}
\begin{euleroutput}
  Function splot2d not found.
  Try list ... to find functions!
  Error in:
  splot2d("normaldis",-3,5): ...
                           ^
\end{euleroutput}
\begin{eulercomment}
Dapat kita lihat dalam kurva fungsi distribusi kumulatif kontinu
terdiri atas tiga bagian yaitu:\\
1. Bernilai 0 untuk x di  bawah minimal dari daerah rentang.\\
2. Merupakan fungsi monoton naik pada daerah rentang.\\
3. Mempunyai nilai konstan 1 di atas batas maksimum daerah rentangnya.

Adapun contoh kurva fungsi distribusi kumulatif diskrit sebagai
berikut.
\end{eulercomment}
\begin{eulerprompt}
>x=normal(1,6);
\end{eulerprompt}
\begin{eulercomment}
Baris kode tersebut akan menghasilkan suatu nilai acak dari distribusi
normal dengan mean 1 dan deviasi standar 6, dan nilai tersebut
disimpan dalam variabel x. Variabel x kemudian dapat digunakan dalam
perhitungan atau analisis selanjutnya

Fungsi empdist(x,vs) membutuhkan array nilai yang diurutkan. Jadi kita
harus mengurutkan x sebelum kita dapat menggunakannya.
\end{eulercomment}
\begin{eulerprompt}
>xs=sort(x);
>plot2d("empdist",-3,5;xs):
\end{eulerprompt}
\eulerimg{25}{images/Kelompok 6_EMT4Statistika-024.png}
\begin{eulercomment}
Grafik fungsi distribusi kumulatif peubah acak diskrit merupakan
fungsi tangga naik dengan nilai terendah 0 dan nilai tertinggi 1.
\end{eulercomment}
\begin{eulercomment}




\begin{eulercomment}
\eulerheading{Sub topik 7 : Menampilkan Tabel Frame Data }
\begin{eulercomment}
Cakupan Materi 1) Diagram Titik\\
2) Diagram Garis\\
3) Kurva Regresi\\
4) Menampilkan Tabel Data Frame

\end{eulercomment}
\eulersubheading{}
\begin{eulercomment}
Diagram Titik Diagram titik atau bisa disebut Scatter Plot adalah tipe
grafik yang digunakan untuk menampilkan nilai-nilai dua variabel pada
sumbu horizontal dan vertikal. Setiap titik dalam diagram mewakili
satu observasi atau data point. Diagram titik sangat berguna untuk
menemukan pola atau hubungan antara dua variabel, serta untuk
mengevaluasi distribusi data.

Dalam scatter plot, sumbu horizontal umumnya digunakan untuk variabel
independen, sementara sumbu vertikal digunakan untuk variabel
dependen. Dengan melihat pola penyebaran titik-titik, kita akan
mendapatkan wawasan tentang apakah ada korelasi antara dua variabel
dan jenis korelasi apa yang mungkin ada (positif, negatif, atau tidak
ada korelasi).

\end{eulercomment}
\begin{eulerprompt}
>x=normal(1,150); plot2d(x,x+rotright(x),>points,style=".."):
\end{eulerprompt}
\eulerimg{25}{images/Kelompok 6_EMT4Statistika-025.png}
\begin{eulercomment}
Diagram titik di atas memvisualisasikan data yang dihasilkan oleh
fungsi normal dengan parameter mean 1 dan standar deviasi 150. Diagram
di atas juga menunjukkan adanya korelasi yang positif karena
pergeseran data tersebut ke kanan.

Berikut adalah penjelasan dari setiap perintah dalam sintaks tersebut:\\
- x=normal(1,150) menghasilkan data dari distribusi normal dengan mean
1 dan standar deviasi 150.\\
- plot2d(x,x+rotright(x),\textgreater{}points,style="..") digunakan untuk membuat
plot 2D dari data x dan x+rotright(x) sebagai sumbu x dan y. \textgreater{}points
digunakan untuk menunjukkan bahwa plot yang dihasilkan berupa
titik-titik, dan style=".." menentukan gaya dari titik-titik tersebut.
\end{eulercomment}
\begin{eulerprompt}
>plot2d(normal(1500),normal(1500),>points,grid=6,style=".."):
\end{eulerprompt}
\eulerimg{25}{images/Kelompok 6_EMT4Statistika-026.png}
\begin{eulercomment}
Diagram titik di atas memvisualisasikan distribusi dari dua set data
yang dihasilkan oleh distribusi normal.

Berikut adalah penjelasan dari setiap perintah dalam sintaks tersebut:\\
plot2d(normal(1500),normal(1500),\textgreater{}points,grid=6,style="..") digunakan
untuk membuat plot 2D dari dua distribusi normal yang menghasilkan
masing-masing 1500 data. \textgreater{}points menunjukkan bahwa plot yang
dihasilkan berupa titik-titik, grid=6 menentukan ukuran grid, dan
style=".." menentukan gaya dari titik-titik tersebut.
\end{eulercomment}
\begin{eulerprompt}
>\{MS,hd\}:=readtable("table1.dat",tok2:=["m","f"]); ...
>writetable(MS,labc=hd,tok2:=["m","f"]);
\end{eulerprompt}
\begin{euleroutput}
      Person       Sex       Age    Mother    Father  Siblings
           1         m        29        58        61         1
           2         f        26        53        54         2
           3         m        24        49        55         1
           4         f        25        56        63         3
           5         f        25        49        53         0
           6         f        23        55        55         2
           7         m        23        48        54         2
           8         m        27        56        58         1
           9         m        25        57        59         1
          10         m        24        50        54         1
          11         f        26        61        65         1
          12         m        24        50        52         1
          13         m        29        54        56         1
          14         m        28        48        51         2
          15         f        23        52        52         1
          16         m        24        45        57         1
          17         f        24        59        63         0
          18         f        23        52        55         1
          19         m        24        54        61         2
          20         f        23        54        55         1
\end{euleroutput}
\begin{eulercomment}
Tabel di atas memvisualisasikan data yang berisi mengenai survei anak,
jenis kelamin mereka, usia mereka, usia orang tua mereka dan jumlah
saudara kandung mereka.

Berikut adalah penjelasan dari setiap perintah dalam sintaks tersebut:\\
- \{MS,hd\}:=readtable("table1.dat",tok2:=["m","f"]) digunakan untuk
membaca data dari file "table1.dat" dan menyimpannya dalam variabel MS
dan hd. tok2:=["m","f"] menunjukkan bahwa data dalam file tersebut
memiliki format "m" dan "f".\\
- writetable(MS,labc=hd,tok2:=["m","f"]) digunakan untuk menulis
kembali data ke file dengan menggunakan label hd sebagai nama kolom
dan format data "m" dan "f".
\end{eulercomment}
\begin{eulerprompt}
>scatterplots(tablecol(MS,3:5),hd[3:5]):
\end{eulerprompt}
\eulerimg{25}{images/Kelompok 6_EMT4Statistika-027.png}
\begin{eulercomment}
Diagram titik di atas  memvisualisasikan hubungan antara tiga variabel
dalam tabel data "MS", yaitu usia anak, ibu, dan ayah.

Berikut adalah penjelasan dari setiap perintah dalam sintaks tersebut:\\
- scatterplots(tablecol(MS,3:5),hd[3:5]) digunakan untuk membuat
scatter plot (diagram titik) dari tiga kolom data dalam tabel MS
dengan menggunakan label hd sebagai nama kolom.\\
- tablecol(MS,3:5) digunakan untuk memilih tiga kolom data dari tabel
MS.\\
- hd[3:5] digunakan untuk memilih tiga label kolom dari hd.

\end{eulercomment}
\eulersubheading{}
\eulersubheading{Contoh Soal}
\begin{eulercomment}
Tabel berikut ini memberikan informasi mengenai kandungan gula (gram)
dan jumlah kalori dalam satu sajian dari 13 sampel merek sereal.
\end{eulercomment}
\begin{eulerprompt}
>DATA:=[ ...
>1,4,120; ...
>2,15,200; ...
>3,12,140; ...
>4,11,110; ...
>5,8,120; ...
>6,6,80; ...
>7,14,170; ...
>8,2,100; ...
>9,7,130; ...
>10,14,190; ...
>11,20,190; ...
>12,3,110; ...
>13,13,120];
>VAR:= ["MERK", "GULA(gram)", "KALORI"];
>writetable(DATA,labc=VAR);
\end{eulerprompt}
\begin{euleroutput}
        MERK GULA(gram)    KALORI
           1          4       120
           2         15       200
           3         12       140
           4         11       110
           5          8       120
           6          6        80
           7         14       170
           8          2       100
           9          7       130
          10         14       190
          11         20       190
          12          3       110
          13         13       120
\end{euleroutput}
\begin{eulercomment}
a. Gambarkan diagram titik atau scatter plot dari data di atas.\\
b. Bagaimana pola penyebaran titik-titik yang telah digambar pada
diagram di atas?\\
c. Kesimpulan seperti apa yang dapat kalian ambil mengenai hubungan
antara gula (gram) dan jumlah kalori?

Penyelesaian:
\end{eulercomment}
\begin{eulerprompt}
>GULA:=[4,15,12,11,8,6,14,2,7,14,20,3,13];
>KALORI:=[120,200,140,110,120,80,170,100,130,190,190,110,120];
>plot2d(GULA,KALORI,>points,color=red,style=".."):
\end{eulerprompt}
\eulerimg{25}{images/Kelompok 6_EMT4Statistika-028.png}
\begin{eulercomment}
a. Diagram titik atau scatter plot di atas merupakan hasil visualisasi
dari tabel data yang telah disebutkan sebelumnya.\\
b. Pada diagram di atas dapat terlihat bahwa pola penyebaran titiknya
mempunyai kecenderungan semakin naik ke atas jika dilihat dari kiri
bawah ke kanan atas.\\
c. Kesimpulan yang dapat diambil yaitu semakin tinggi kandungan gula
maka semakin tinggi jumlah kalorinya.
\end{eulercomment}
\eulersubheading{}
\begin{eulercomment}
Diagram Garis 
Diagram garis atau bisa disebut line chart adalah jenis grafik
statistik yang digunakan untuk menunjukkan perubahan atau tren
sepanjang waktu atau variabel independen lainnya. Ini adalah salah
satu cara yang paling umum digunakan untuk memvisualisasikan data
berurutan dalam statistika.

Dalam diagram garis, data diplotkan sebagai titik-titik dan kemudian
dihubungkan dengan garis lurus. Ini membantu untuk melihat perubahan
atau tren dari satu titik ke titik berikutnya dengan jelas. Diagram
garis sangat berguna untuk menyoroti pola, tren, atau fluktuasi dalam
data sepanjang periode waktu tertentu.

Contoh umum penggunaan diagram garis melibatkan waktu di sumbu
horizontal (x) dan nilai atau frekuensi di sumbu vertikal (y).
Misalnya, diagram garis dapat digunakan untuk menunjukkan perubahan
suhu sepanjang waktu, pertumbuhan populasi, atau kinerja keuangan
perusahaan sepanjang beberapa kuartal.

Untuk memetakan data, kita mencoba hasil pemilu Jerman sejak tahun
1990, diukur dalam jumlah kursi.
\end{eulercomment}
\begin{eulerprompt}
>BW := [...
>1990,662,319,239,79,8,17; ...
>1994,672,294,252,47,49,30; ...
>1998,669,245,298,43,47,36; ...
>2002,603,248,251,47,55,2; ...
>2005,614,226,222,61,51,54; ...
>2009,622,239,146,93,68,76; ...
>2013,631,311,193,0,63,64];
\end{eulerprompt}
\begin{eulercomment}
Sintaks tersebut digunakan untuk membuat matriks BW yang berisi
data-data tahun dan jumlah kursi untuk beberapa partai politik pada
setiap tahun.
\end{eulercomment}
\begin{eulerprompt}
>P:=["CDU/CSU","SPD","FDP","Gr","Li"];
\end{eulerprompt}
\begin{eulercomment}
Sintaks tersebut digunakan untuk membuat array P yang berisi
label-label untuk setiap kolom pada matriks BW.
\end{eulercomment}
\begin{eulerprompt}
>BT:=BW[,3:7]; BT:=BT/sum(BT); YT:=BW[,1]';
\end{eulerprompt}
\begin{eulercomment}
Berikut adalah penjelasan dari setiap perintah dalam sintaks tersebut:\\
- BT:=BW[,3:7]; BT:=BT/sum(BT) digunakan untuk mengambil kolom ke-3
hingga ke-7 dari matriks BW, kemudian membagi setiap nilai dalam
matriks tersebut dengan jumlah total nilai dalam matriks tersebut.
Kolom 3 sampai 7 adalah jumlah kursi masing-masing partai, dan kolom 2
adalah jumlah kursi seluruhnya. Sedangkan kolom 1 adalah tahun
pemilihan.\\
- YT:=BW[,1]' digunakan untuk mengambil transpose dari kolom pertama
matriks BW dan menyimpannya dalam variabel YT.
\end{eulercomment}
\begin{eulerprompt}
>writetable(BT*100,wc=6,dc=0,>fixed,labc=P,labr=YT)
\end{eulerprompt}
\begin{euleroutput}
         CDU/CSU   SPD   FDP    Gr    Li
    1990      48    36    12     1     3
    1994      44    38     7     7     4
    1998      37    45     6     7     5
    2002      41    42     8     9     0
    2005      37    36    10     8     9
    2009      38    23    15    11    12
    2013      49    31     0    10    10
\end{euleroutput}
\begin{eulercomment}
Sintaks tersebut digunakan untuk menulis matriks BT yang telah
dikalikan dengan 100 ke dalam sebuah file. Parameter wc=6 menunjukkan
lebar atau sekat kolom, dc=0 menunjukkan jumlah digit di belakang
koma, \textgreater{}fixed menunjukkan bahwa jumlah digit di belakang koma tetap,
labc=P menunjukkan label untuk kolom, dan labr=YT menunjukkan label
untuk baris.
\end{eulercomment}
\begin{eulerprompt}
>BT1:=(BT.[1;1;0;0;0])'*100
\end{eulerprompt}
\begin{euleroutput}
  [84.29,  81.25,  81.1659,  82.7529,  72.9642,  61.8971,  79.8732]
\end{euleroutput}
\begin{eulercomment}
Sintaks di atas digunakan untuk mengalikan baris pertama dari matriks
BT dengan 100 dan menyimpan hasilnya dalam variabel BT1.  Angka-angka
dalam tanda kurung siku menunjukkan indeks baris yang ingin diambil
dari matriks BT (pada sintaks tersebut indeks baris yang ingin diambil
yaitu baris pertama), sedangkan tanda kutip di sebelah kanan
menunjukkan bahwa kita ingin mengalikan baris tersebut dengan 100.
\end{eulercomment}
\begin{eulerprompt}
>statplot(YT,BT1,"b"):
\end{eulerprompt}
\eulerimg{25}{images/Kelompok 6_EMT4Statistika-029.png}
\begin{eulercomment}
Diagram di atas memvisualisasikan persentase perolehan jumlah kursi
setiap partai di setiap tahunnya, yaitu dari tahun 1990-2013.

Tipe plotplot yang tersedia adalah sebagai berikut:\\
- 'p' : plot titik\\
- 'l' : plot garis\\
- 'b' : keduanya (titik dan garis)\\
- 'h' : alur histogram\\
- 's' : plot permukaan
\end{eulercomment}
\eulersubheading{}
\eulersubheading{Contoh Soal}
\begin{eulercomment}
Perhatikan tabel di bawah ini!
\end{eulercomment}
\begin{eulerprompt}
>BARANG:=[ ...
>2015,13000,10000,2700,10700; ...
>2020,15000,12500,4000,12300];
>VRB := ["TAHUN","SUSU(kaleng)","GULA(kg)","JAGUNG(kg)","BERAS(kg)"];
>writetable(BARANG,labc=VRB);
\end{eulerprompt}
\begin{euleroutput}
       TAHUN SUSU(kaleng)  GULA(kg) JAGUNG(kg) BERAS(kg)
        2015        13000     10000       2700     10700
        2020        15000     12500       4000     12300
\end{euleroutput}
\begin{eulercomment}
Hitunglah persentase kenaikan harga dari tahun 2015 hingga tahun 2020
(jika penghitungan indeks harga menggunakan metode agregatif sedehana)
dan gambarkan grafik kenaikan harganya menggunakan diagram garis!

Penyelesaian:
\end{eulercomment}
\begin{eulerprompt}
>HARGA:=BARANG[,2:5]; TAHUN:=BARANG[,1]';
>Po:=(13000+10000+2700+10700) // total harga barang di tahun 2015
\end{eulerprompt}
\begin{euleroutput}
  36400
\end{euleroutput}
\begin{eulerprompt}
>Pn:=(15000+12500+4000+12300) // total harga barang di tahun 2020
\end{eulerprompt}
\begin{euleroutput}
  43800
\end{euleroutput}
\begin{eulerprompt}
>Persentase:=(Pn/Po)*100
\end{eulerprompt}
\begin{euleroutput}
  120.32967033
\end{euleroutput}
\begin{eulerprompt}
>Kenaikan:=(120.3-100)
\end{eulerprompt}
\begin{euleroutput}
  20.3
\end{euleroutput}
\begin{eulercomment}
Berdasarkan penghitungan tersebut, dapat disimpulkan bahwa dari tahun
2015 ke tahun 2020 terdapat kenaikan harga sebesar 20,3\%.
\end{eulercomment}
\begin{eulerprompt}
>P1:=[36400,43800];
>statplot(TAHUN,P1,"b"):
\end{eulerprompt}
\eulerimg{25}{images/Kelompok 6_EMT4Statistika-030.png}
\begin{eulercomment}
Diagram garis di atas memvisualisasikan kenaikan harga barang dari
tahun 2015 hingga tahun 2020.
\end{eulercomment}
\begin{eulercomment}

\end{eulercomment}
\eulersubheading{}
\begin{eulercomment}
Kurva Regresi 
Kurva regresi adalah representasi grafis dari model regresi yang
digunakan untuk memodelkan hubungan antara variabel independen (X) dan
variabel dependen (Y). Persamaan regresi dapat digunakan untuk
memprediksi atau mengestimasi nilai dari variabel tak bebas
berdasarkan informasi dari variabel bebas. Persamaan regresi linear
merupakan suatu persamaan yang berupa garis lurus, sedangkan persamaan
regresi nonlinear bukan merupakan persamaan garis lurus.

Persamaan regresi linier umumnya ditulis sebagai Y=a+bX, di mana: \\
Y = garis regresi/ variable response\\
a = konstanta (intercept), perpotongan dengan sumbu-y\\
b = konstanta regresi (slope), kemiringan\\
X = variabel bebas/ predictor

Regresi linier dapat dilakukan dengan fungsi polyfit() atau berbagai
fungsi fit. Sebagai permulaan, kami menemukan garis regresi untuk data
univariat dengan polyfit(x,y,1).

\end{eulercomment}
\begin{eulerprompt}
>x=1:10; y=[2,3,1,5,6,3,7,8,9,8]; writetable(x'|y',labc=["x","y"])
\end{eulerprompt}
\begin{euleroutput}
           x         y
           1         2
           2         3
           3         1
           4         5
           5         6
           6         3
           7         7
           8         8
           9         9
          10         8
\end{euleroutput}
\begin{eulercomment}
Berikut adalah penjelasan dari setiap perintah dalam sintaks tersebut:\\
- x=1:10; y=[2,3,1,5,6,3,7,8,9,8];: Perintah ini digunakan untuk
membuat vektor x yang berisi bilangan bulat dari 1 hingga 10, dan
vektor y yang berisi nilai-nilai acak.\\
- writetable(x'\textbar{}y',labc=["x","y"]): Perintah ini digunakan untuk
membuat tabel dua kolom dari dua vektor x dan y. Parameter x'\textbar{}y'
menunjukkan bahwa vektor x dan y akan disusun secara vertikal,
sedangkan labc=["x","y"] mengatur label kolom tabel.

Kami ingin membandingkan kecocokan yang tidak berbobot dan berbobot.
Pertama koefisien kecocokan linier.
\end{eulercomment}
\begin{eulerprompt}
>p=polyfit(x,y,1)
\end{eulerprompt}
\begin{euleroutput}
  [0.733333,  0.812121]
\end{euleroutput}
\begin{eulercomment}
Sintaks di atas digunakan untuk melakukan fitting kurva polinomial
orde satu pada data yang diberikan. Parameter x dan y adalah data yang
akan di-fit, sedangkan 1 menunjukkan orde polinomial yang digunakan.
Hasil fitting akan disimpan dalam variabel p.

Sekarang koefisien dengan bobot yang menekankan nilai terakhir.
\end{eulercomment}
\begin{eulerprompt}
>w &= "exp(-(x-10)^2/10)"; pw=polyfit(x,y,1,w=w(x))
\end{eulerprompt}
\begin{euleroutput}
  [4.71566,  0.38319]
\end{euleroutput}
\begin{eulercomment}
Berikut adalah penjelasan dari setiap perintah dalam sintaks tersebut:\\
- w \&= "exp(-(x-10)\textasciicircum{}2/10)": digunakan untuk mendefinisikan fungsi
berat (weight function) w sebagai exp(-(x-10)\textasciicircum{}2/10). Fungsi berat
digunakan dalam fitting polinomial untuk memberikan bobot pada setiap
titik data, sehingga titik-titik data yang lebih dekat dengan nilai 10
akan diberi bobot yang lebih tinggi.\\
- pw=polyfit(x,y,1,w=w(x)): digunakan untuk melakukan fitting
polinomial pada data yang diberikan dengan menggunakan fungsi berat w.
Fungsi polyfit digunakan untuk menemukan koefisien polinomial yang
sesuai dengan data. Dalam kasus ini, fitting dilakukan dengan
polinomial derajat 1, dengan menggunakan fungsi berat yang telah
didefinisikan sebelumnya.

Kami memasukkan semuanya ke dalam satu plot untuk titik dan garis
regresi, dan untuk bobot yang digunakan.
\end{eulercomment}
\begin{eulerprompt}
>figure(2,1);  ...
>figure(1); statplot(x,y,"b",xl="Regression"); ...
>  plot2d("evalpoly(x,p)",>add,color=blue,style="--"); ...
>  plot2d("evalpoly(x,pw)",5,10,>add,color=red,style="--"); ...
>figure(2); plot2d(w,1,10,>filled,style="/",fillcolor=red,xl=w); ...
>figure(0):
\end{eulerprompt}
\eulerimg{25}{images/Kelompok 6_EMT4Statistika-031.png}
\begin{eulercomment}
Berikut adalah penjelasan dari setiap perintah dalam sintaks tersebut:\\
- figure(2,1); digunakan untuk mengatur tampilan grafik dengan berbeda
nomor tampilan (handle).\\
- statplot(x,y,"b",xl="Regression"): digunakan untuk menampilkan data
(x,y) dengan label nama sumbu-x nya yaitu "Regression".\\
- plot2d("evalpoly(x,p)",\textgreater{}add,color=blue,style="--"): fungsi evalpoly
digunakan untuk menghitung nilai polinomial pada titik-titik data yang
diberikan. Dalam hal ini, evalpoly(x,p) menghitung nilai polinomial
pada titik-titik data yang diberikan dari polinomial pertama (p).
Hasilnya kemudian digunakan untuk menambahkan plot polinomial pertama
ke grafik dengan warna biru dan gaya putus-putus.\\
- plot2d("evalpoly(x,pw)",5,10,\textgreater{}add,color=red,style="--"): fungsi
evalpoly digunakan untuk menghitung nilai polinomial yang diperoleh
dari polyfit dengan fungsi berat. Dalam hal ini, evalpoly(x,pw)
menghitung nilai polinomial dengan fungsi berat pada titik-titik data
dari 5 hingga 10. Hasilnya kemudian digunakan untuk menambahkan plot
polinomial dengan fungsi berat ke grafik dengan warna merah dan gaya
putus-putus.\\
- plot2d(w,1,10,\textgreater{}filled,style="/",fillcolor=red,xl=w): sintaks
tersebut  digunakan untuk membuat plot dari area di bawah kurva
polinomial dengan fungsi berat w pada rentang 1 hingga 10, dengan gaya
pengisian warna merah dan label sumbu x yang diberi nilai w
\end{eulercomment}
\begin{eulercomment}

\end{eulercomment}
\eulersubheading{}
\begin{eulercomment}
Menampilkan Data Frame 
Data frame biasanya merujuk pada struktur data tabular dua dimensi
yang digunakan untuk menyimpan dan mengorganisir data. Data frame
adalah konsep yang umumnya terkait dengan pemrograman statistik,
terutama dalam bahasa seperti R dan Python (menggunakan pustaka
pandas).

Di direktori buku catatan ini kita akan menemukan file dengan tabel.
Data tersebut merupakan hasil survei. Berikut adalah empat baris
pertama file tersebut. Datanya berasal dari buku online Jerman
"Einführung in die Statistik mit R" yang dibuat oleh A. Handl.
\end{eulercomment}
\begin{eulerprompt}
>printfile("table.dat",4);
\end{eulerprompt}
\begin{euleroutput}
  Person Sex Age Titanic Evaluation Tip Problem
  1 m 30 n . 1.80 n
  2 f 23 y g 1.80 n
  3 f 26 y g 1.80 y
\end{euleroutput}
\begin{eulercomment}
Sintaks tersebut digunakan untuk mencetak isi file "table.dat" pada
empat baris pertama.

Tabel berisi 7 kolom angka atau token (string). Kami ingin membaca
tabel dari file. Pertama, kami menggunakan terjemahan kami sendiri
untuk token.

Untuk ini, kami mendefinisikan set token. Fungsi strtokens()
mendapatkan vektor string token dari string yang diberikan.
\end{eulercomment}
\begin{eulerprompt}
>mf:=["m","f"]; yn:=["y","n"]; ev:=strtokens("g vg m b vb");
\end{eulerprompt}
\begin{eulercomment}
Berikut adalah penjelasan dari setiap perintah dalam sintaks tersebut:\\
- mf:=["m","f"]: Mendefinisikan variabel mf sebagai array yang berisi
dua elemen, yaitu "m" dan "f". Kemungkinan ini digunakan untuk
merepresentasikan jenis kelamin, di mana "m" mewakili laki-laki dan
"f" mewakili perempuan.\\
- yn:=["y","n"]: Mendefinisikan variabel yn sebagai array yang berisi
dua elemen, yaitu "y" dan "n". Kemungkinan ini digunakan untuk
merepresentasikan jawaban ya ("y") dan tidak ("n") dari suatu
pertanyaan.\\
- ev:=strtokens("g vg m b vb"): Mendefinisikan variabel ev sebagai
hasil dari pemisahan string "g vg m b vb" berdasarkan spasi.
Kemungkinan ini digunakan untuk merepresentasikan kategori tertentu
yang terkait dengan data atau survei.

Sekarang kita membaca tabel dengan terjemahan ini.

Argumen tok2, tok4 dll. adalah terjemahan dari kolom tabel. Argumen
ini tidak ada dalam daftar parameter readtable(), jadi Anda harus
menyediakannya dengan ":=".
\end{eulercomment}
\begin{eulerprompt}
>\{MT,hd\}=readtable("table.dat",tok2:=mf,tok4:=yn,tok5:=ev,tok7:=yn);
\end{eulerprompt}
\begin{eulercomment}
Sintaks ini digunakan untuk membaca data dari file "table.dat" ke
dalam sebuah tabel. Pada saat membaca file, argumen opsional tok2,
tok4, tok5, dan tok7 digunakan untuk menentukan bagaimana data dalam
file tersebut akan diinterpretasikan.
\end{eulercomment}
\begin{eulerprompt}
>load over statistics;
\end{eulerprompt}
\begin{eulercomment}
Sintaks ini digunakan untuk memuat data atau variabel yang telah
disimpan sebelumnya.

Untuk mencetak, kita perlu menentukan set token yang sama. Kami
mencetak empat baris pertama saja.
\end{eulercomment}
\begin{eulerprompt}
>writetable(MT[1:4],labc=hd,wc=6,tok2:=mf,tok4:=yn,tok5:=ev,tok7:=yn);
\end{eulerprompt}
\begin{euleroutput}
   Person   Sex   Age Titanic Evaluation   Tip Problem
        1     m    30       n          .   1.8       n
        2     f    23       y          g   1.8       n
        3     f    26       y          g   1.8       y
        4     m    33       n          .   2.8       n
\end{euleroutput}
\begin{eulercomment}
Titik "." mewakili nilai-nilai, yang tidak tersedia.

Jika kita tidak ingin menentukan token untuk terjemahan terlebih
dahulu, kita hanya perlu menentukan, kolom mana yang berisi token dan
bukan angka.
\end{eulercomment}
\begin{eulerprompt}
>ctok=[2,4,5,7]; \{MT,hd,tok\}=readtable("table.dat",ctok=ctok);
\end{eulerprompt}
\begin{eulercomment}
Fungsi readtable() sekarang mengembalikan satu set token.

Berikut adalah penjelasan dari setiap perintah dalam sintaks tersebut:\\
- ctok=[2,4,5,7]: digunakan untuk mendefinisikan variabel ctok sebagai
array yang berisi angka 2, 4, 5, dan 7. Kemungkinan ini digunakan
untuk menentukan kolom-kolom tertentu yang akan dibaca dari file
"table.dat".\\
- \{MT,hd,tok\}=readtable("table.dat",ctok=ctok): sintaks ini
menggunakan fungsi readtable untuk membaca data dari file "table.dat"
ke dalam tiga variabel yang berbeda, yaitu MT, hd, dan tok, dengan
menggunakan kolom-kolom yang telah ditentukan sebelumnya dalam
variabel ctok.
\end{eulercomment}
\begin{eulerprompt}
>tok
\end{eulerprompt}
\begin{euleroutput}
  m
  n
  f
  y
  g
  vg
\end{euleroutput}
\begin{eulercomment}
Tabel berisi entri dari file dengan token yang diterjemahkan ke angka.

String khusus NA="." ditafsirkan sebagai "Tidak Tersedia", dan
mendapatkan NAN (bukan angka) dalam tabel. Terjemahan ini dapat diubah
dengan parameter NA, dan NAval.
\end{eulercomment}
\begin{eulerprompt}
>MT[1]
\end{eulerprompt}
\begin{euleroutput}
  [1,  1,  30,  2,  NAN,  1.8,  2]
\end{euleroutput}
\begin{eulercomment}
Sintaks tersebut digunakan untuk mengakses elemen pertama dari
variabel MT yang telah diimpor sebelumnya.

Berikut isi tabel dengan nomor yang belum diterjemahkan.
\end{eulercomment}
\begin{eulerprompt}
>writetable(MT,wc=5)
\end{eulerprompt}
\begin{euleroutput}
      1    1   30    2    .  1.8    2
      2    3   23    4    5  1.8    2
      3    3   26    4    5  1.8    4
      4    1   33    2    .  2.8    2
      5    1   37    2    .  1.8    2
      6    1   28    4    5  2.8    4
      7    3   31    4    6  2.8    2
      8    1   23    2    .  0.8    2
      9    3   24    4    6  1.8    4
     10    1   26    2    .  1.8    2
     11    3   23    4    6  1.8    4
     12    1   32    4    5  1.8    2
     13    1   29    4    6  1.8    4
     14    3   25    4    5  1.8    4
     15    3   31    4    5  0.8    2
     16    1   26    4    5  2.8    2
     17    1   37    2    .  3.8    2
     18    1   38    4    5    .    2
     19    3   29    2    .  3.8    2
     20    3   28    4    6  1.8    2
     21    3   28    4    1  2.8    4
     22    3   28    4    6  1.8    4
     23    3   38    4    5  2.8    2
     24    3   27    4    1  1.8    4
     25    1   27    2    .  2.8    4
\end{euleroutput}
\begin{eulercomment}
Untuk kenyamanan, Anda dapat memasukkan keluaran readtable() ke dalam
list.
\end{eulercomment}
\begin{eulerprompt}
>Table=\{\{readtable("table.dat",ctok=ctok)\}\};
\end{eulerprompt}
\begin{eulercomment}
Dengan menggunakan kolom token yang sama dan token yang dibaca dari
file, kita dapat mencetak tabel. Kita dapat menentukan ctok, tok, dll
atau menggunakan tabel daftar
\end{eulercomment}
\begin{eulerprompt}
>writetable(Table,ctok=ctok,wc=5);
\end{eulerprompt}
\begin{euleroutput}
   Person  Sex  Age Titanic Evaluation  Tip Problem
        1    m   30       n          .  1.8       n
        2    f   23       y          g  1.8       n
        3    f   26       y          g  1.8       y
        4    m   33       n          .  2.8       n
        5    m   37       n          .  1.8       n
        6    m   28       y          g  2.8       y
        7    f   31       y         vg  2.8       n
        8    m   23       n          .  0.8       n
        9    f   24       y         vg  1.8       y
       10    m   26       n          .  1.8       n
       11    f   23       y         vg  1.8       y
       12    m   32       y          g  1.8       n
       13    m   29       y         vg  1.8       y
       14    f   25       y          g  1.8       y
       15    f   31       y          g  0.8       n
       16    m   26       y          g  2.8       n
       17    m   37       n          .  3.8       n
       18    m   38       y          g    .       n
       19    f   29       n          .  3.8       n
       20    f   28       y         vg  1.8       n
       21    f   28       y          m  2.8       y
       22    f   28       y         vg  1.8       y
       23    f   38       y          g  2.8       n
       24    f   27       y          m  1.8       y
       25    m   27       n          .  2.8       y
\end{euleroutput}
\begin{eulercomment}
Fungsi tablecol() mengembalikan nilai kolom tabel, melewatkan baris
apa pun dengan nilai NAN ("." dalam file), dan indeks kolom, yang
berisi nilai-nilai ini.
\end{eulercomment}
\begin{eulerprompt}
>\{c,i\}=tablecol(MT,[5,6]);
\end{eulerprompt}
\begin{eulercomment}
Sintaks ini digunakan untuk mengekstrak kolom ke-5 dan ke-6 dari
matriks atau tabel MT. Hasilnya akan disimpan dalam variabel c dan i
secara berturut-turut. Dengan kata lain, c akan berisi kolom ke-5 dari
MT, sedangkan i akan berisi kolom ke-6 dari MT.

Kita bisa menggunakan ini untuk mengekstrak kolom dari tabel untuk
tabel baru.
\end{eulercomment}
\begin{eulerprompt}
>j=[1,5,6]; writetable(MT[i,j],labc=hd[j],ctok=[2],tok=tok)
\end{eulerprompt}
\begin{euleroutput}
      Person Evaluation       Tip
           2          g       1.8
           3          g       1.8
           6          g       2.8
           7         vg       2.8
           9         vg       1.8
          11         vg       1.8
          12          g       1.8
          13         vg       1.8
          14          g       1.8
          15          g       0.8
          16          g       2.8
          20         vg       1.8
          21          m       2.8
          22         vg       1.8
          23          g       2.8
          24          m       1.8
\end{euleroutput}
\begin{eulercomment}
Berikut adalah maksud dari sintaks tersebut:\\
- MT[i,j]: Mengambil subset dari tabel MT yang terdiri dari
baris-baris yang ditentukan oleh i dan kolom-kolom yang ditentukan
oleh j. Ini mungkin digunakan untuk membuat subset dari data yang akan
ditulis ke dalam file.\\
- labc=hd[j]: Menentukan label kolom untuk subset kolom yang dipilih
dari MT berdasarkan label kolom dari hd. Ini mungkin digunakan untuk
menetapkan label yang sesuai untuk kolom-kolom yang akan ditulis ke
dalam file.\\
- ctok=[2]: Menentukan token tertentu yang akan digunakan saat menulis
tabel ke dalam file. Ini mungkin digunakan untuk menetapkan cara
khusus untuk menginterpretasikan data saat ditulis ke dalam file.\\
- tok=tok: Menggunakan token yang telah ditentukan sebelumnya saat
menulis tabel ke dalam file. Ini mungkin digunakan untuk memastikan
bahwa token yang sama digunakan saat menulis dan membaca tabel.

Tentu saja, kita perlu mengekstrak tabel itu sendiri dari daftar Tabel
dalam kasus ini.
\end{eulercomment}
\begin{eulerprompt}
>MT=Table[1];
\end{eulerprompt}
\begin{eulercomment}
Tentu saja, kita juga dapat menggunakannya untuk menentukan nilai
rata-rata suatu kolom atau nilai statistik lainnya.
\end{eulercomment}
\begin{eulerprompt}
>mean(tablecol(MT,6))
\end{eulerprompt}
\begin{euleroutput}
  2.175
\end{euleroutput}
\begin{eulercomment}
Fungsi getstatistics() mengembalikan elemen dalam vektor, dan
jumlahnya. 
\end{eulercomment}
\begin{eulerprompt}
>\{xu,count\}=getstatistics(tablecol(MT,5)); xu, count,
\end{eulerprompt}
\begin{euleroutput}
  [1,  5,  6]
  [2,  9,  6]
\end{euleroutput}
\begin{eulercomment}
Variabel xu adalah elemen unik (indeks) dari kolom ke-5 tabel MT,
sedangkan variabel count adalah jumlah data dari variabel xu.

Kita bisa mencetak hasilnya di tabel baru.
\end{eulercomment}
\begin{eulerprompt}
>writetable(count',labr=tok[xu])
\end{eulerprompt}
\begin{euleroutput}
           m         2
           g         9
          vg         6
\end{euleroutput}
\begin{eulercomment}
Fungsi selecttable() mengembalikan tabel baru dengan nilai dalam satu
kolom yang dipilih dari vektor indeks.\\
Pertama kita mencari indeks dari dua nilai kita di tabel token.
\end{eulercomment}
\begin{eulerprompt}
>v:=indexof(tok,["g","vg"])
\end{eulerprompt}
\begin{euleroutput}
  [5,  6]
\end{euleroutput}
\begin{eulercomment}
Sekarang kita dapat memilih baris tabel, yang memiliki salah satu
nilai v pada baris ke-5.
\end{eulercomment}
\begin{eulerprompt}
>MT1:=MT[selectrows(MT,5,v)]; i:=sortedrows(MT1,5);
\end{eulerprompt}
\begin{eulercomment}
Sekarang kita dapat mencetak tabel, dengan nilai yang diekstraksi dan
diurutkan di kolom ke-5.
\end{eulercomment}
\begin{eulerprompt}
>writetable(MT1[i],labc=hd,ctok=ctok,tok=tok,wc=7);
\end{eulerprompt}
\begin{euleroutput}
   Person    Sex    Age Titanic Evaluation    Tip Problem
        2      f     23       y          g    1.8       n
        3      f     26       y          g    1.8       y
        6      m     28       y          g    2.8       y
       18      m     38       y          g      .       n
       16      m     26       y          g    2.8       n
       15      f     31       y          g    0.8       n
       12      m     32       y          g    1.8       n
       23      f     38       y          g    2.8       n
       14      f     25       y          g    1.8       y
        9      f     24       y         vg    1.8       y
        7      f     31       y         vg    2.8       n
       20      f     28       y         vg    1.8       n
       22      f     28       y         vg    1.8       y
       13      m     29       y         vg    1.8       y
       11      f     23       y         vg    1.8       y
\end{euleroutput}
\begin{eulercomment}
Untuk statistik selanjutnya, kami ingin menghubungkan dua kolom tabel.
Jadi kita ekstrak kolom 2 dan 4 dan urutkan tabelnya.
\end{eulercomment}
\begin{eulerprompt}
>i=sortedrows(MT,[2,4]); ...
>writetable(tablecol(MT[i],[2,4])',ctok=[1,2],tok=tok)
\end{eulerprompt}
\begin{euleroutput}
           m         n
           m         n
           m         n
           m         n
           m         n
           m         n
           m         n
           m         y
           m         y
           m         y
           m         y
           m         y
           f         n
           f         y
           f         y
           f         y
           f         y
           f         y
           f         y
           f         y
           f         y
           f         y
           f         y
           f         y
           f         y
\end{euleroutput}
\begin{eulercomment}
Dengan getstatistics(), kita juga bisa menghubungkan jumlah dalam dua
kolom tabel satu sama lain.
\end{eulercomment}
\begin{eulerprompt}
>MT24=tablecol(MT,[2,4]); ...
>\{xu1,xu2,count\}=getstatistics(MT24[1],MT24[2]); ...
>writetable(count,labr=tok[xu1],labc=tok[xu2])
\end{eulerprompt}
\begin{euleroutput}
                     n         y
           m         7         5
           f         1        12
\end{euleroutput}
\begin{eulercomment}
Sebuah tabel dapat ditulis ke file.
\end{eulercomment}
\begin{eulerprompt}
>filename="test.dat"; ...
>writetable(count,labr=tok[xu1],labc=tok[xu2],file=filename);
\end{eulerprompt}
\begin{eulercomment}
Kemudian kita bisa membaca tabel dari file tersebut.
\end{eulercomment}
\begin{eulerprompt}
>\{MT2,hd,tok2,hdr\}=readtable(filename,>clabs,>rlabs); ...
>writetable(MT2,labr=hdr,labc=hd)
\end{eulerprompt}
\begin{euleroutput}
                     n         y
           m         7         5
           f         1        12
\end{euleroutput}
\begin{eulercomment}
Dan hapus file tersebut.
\end{eulercomment}
\begin{eulerprompt}
>fileremove(filename);
\end{eulerprompt}
\begin{eulercomment}
\begin{eulercomment}
\eulerheading{Sub Topik 8 : Melakukan Perhitungan Untuk Uji Statistika}
\begin{eulercomment}
Perhitungan untuk Uji Statistika Dalam Euler, banyak uji yang
diterapkan. Semua uji dalam Euler mengembalikan kesalahan yang
diterima jika hipotesis nol ditolak.

Setiap jenis uji statistik memiliki asumsi dan syarat tertentu yang
harus dipenuhi sebelum dapat diterapkan pada data. Pemilihan jenis uji
statistik yang tepat sangat penting untuk memastikan hasil analisis
data yang akurat dan dapat diandalkan.

\end{eulercomment}
\eulersubheading{Simulasi Monte Carlo}
\begin{eulercomment}
Simulasi Monte Carlo yang digunakan untuk memecahkan masalah dengan
menggunakan sampel acak berulang untuk memperkirakan distribusi dari
suatu fenomena. Dalam konteks uji statistika, Monte Carlo dapat
digunakan untuk menghitung nilai-nilai yang sulit atau tidak mungkin
dihitung secara analitik, seperti p-value dalam uji chi-square.

Euler dapat digunakan untuk mensimulasikan kejadian acak. Berikut ini
contoh yang mensimulasikan 1000 kali 3 lemparan dadu, dan ditampilkan
distribusi jumlahnya.

Fungsi ini akan menghasilkan 1000 bilangan bulat acak antara 3 dan 6,
menjumlahkannya, dan menyimpan hasilnya dalam ds.

Fungsi getmultiplicities(v,x) mengembalikan kelipatan dari nilai-nilai
dalam v dalam vektor x.\\
Dalam hal ini, fs akan berisi kelipatan dari nilai 3 hingga 18 dalam
vektor ds.
\end{eulercomment}
\begin{eulerprompt}
>ds:=sum(intrandom(1000,3,6))';  fs=getmultiplicities(3:18,ds)
\end{eulerprompt}
\begin{euleroutput}
  [5,  15,  31,  42,  75,  98,  132,  114,  122,  106,  83,  74,  54,
  25,  16,  8]
\end{euleroutput}
\begin{eulerprompt}
>columnsplot(fs,lab=3:18):
\end{eulerprompt}
\eulerimg{25}{images/Kelompok 6_EMT4Statistika-032.png}
\begin{eulercomment}
Untuk menentukan distribusi yang diharapkan tidak begitu mudah.
Diperlukan rekursi lanjutan untuk hal ini.

Fungsi berikut menghitung banyaknya cara bilangan k dapat dinyatakan
sebagai jumlah dari n bilangan dalam rentang 1 sampai m. Hal berikut
bekerja secara rekursif dengan cara yang jelas.
\end{eulercomment}
\begin{eulerprompt}
>function map countways (k; n, m) ...
\end{eulerprompt}
\begin{eulerudf}
    if n==1 then return k>=1 && k<=m
    else
      sum=0; 
      loop 1 to m; sum=sum+countways(k-#,n-1,m); end;
      return sum;
    end;
  endfunction
\end{eulerudf}
\begin{eulerprompt}
>cw=countways(3:18,3,6)
\end{eulerprompt}
\begin{euleroutput}
  [1,  3,  6,  10,  15,  21,  25,  27,  27,  25,  21,  15,  10,  6,  3,
  1]
\end{euleroutput}
\begin{eulercomment}
Nilai yang diharapkan dapat direpresentasikan ke dalam plot.\\
Fungsi countways menghitung jumlah cara di mana bilangan k dapat
direpresentasikan sebagai jumlah dari n bilangan dalam rentang 1
hingga m.

Fungsi ini bekerja secara rekursif dengan cara yang jelas. Jika n=1,
maka fungsi mengembalikan k\textgreater{}=1 \&\& k\textless{}=m, jika tidak, fungsi akan
menghitung jumlah cara dengan melakukan iterasi dari 1 hingga m.
Hasilnya adalah jumlah cara di mana bilangan k dapat direpresentasikan
sebagai jumlah dari n bilangan dalam rentang 1 hingga m.
\end{eulercomment}
\begin{eulerprompt}
>plot2d(cw/6^3*1000,>add); plot2d(cw/6^3*1000,>points,>add):
\end{eulerprompt}
\eulerimg{25}{images/Kelompok 6_EMT4Statistika-033.png}
\begin{eulercomment}
\end{eulercomment}
\eulersubheading{Uji Chi Square}
\begin{eulercomment}
Untuk melakukan uji chi\textasciicircum{}2 (chi-squared) kita menggunakan fungsi
chitest(). Fungsi ini digunakan untuk membandingkan frekuensi
observasi dengan frekuensi yang diharapkan. Dalam konteks ini, vektor
x merepresentasikan frekuensi observasi, sedangkan vektor y
merepresentasikan frekuensi yang diharapkan. Misalnya, jika dari
sampel 100 orang ditemukan 40 pria, maka vektor observasi x adalah
[40,60], dan vektor harapan y mungkin [50,50]. Uji chi\textasciicircum{}2 digunakan
untuk menilai sejauh mana sampel tersebut sesuai dengan frekuensi yang
diharapkan.

Sebagai contoh, kami menguji lemparan dadu untuk distribusi baju. Pada
600 lemparan, kami mendapatkan nilai berikut, yang kami masukkan ke
dalam uji chi-kuadrat.
\end{eulercomment}
\begin{eulerprompt}
>chitest([90,103,114,101,103,89],dup(100,6)')
\end{eulerprompt}
\begin{euleroutput}
  0.498830517952
\end{euleroutput}
\begin{eulercomment}
Fungsi ini digunakan untuk menguji asosiasi antara dua set data. Set
pertama [90,103,114,101,103,89] mewakili jumlah yang diamati, dan set
kedua dup(100,6) mewakili jumlah yang diharapkan. Hasil dari fungsi
chitest adalah probabilitas yang terkait dengan statistik uji
chi-kuadrat, yang menunjukkan kemungkinan bahwa data kategoris yang
diamati diambil dari distribusi yang diharapkan

Tes chi-kuadrat juga memiliki mode, yang menggunakan simulasi Monte
Carlo untuk menguji statistik. Hasilnya harus hampir sama. Parameter
\textgreater{}p menginterpretasikan vektor-y sebagai vektor probabilitas.
\end{eulercomment}
\begin{eulerprompt}
>chitest([90,103,114,101,103,89],dup(1/6,6)',>p,>montecarlo)
\end{eulerprompt}
\begin{euleroutput}
  0.497
\end{euleroutput}
\begin{eulercomment}
Kesalahan ini terlalu besar. Jadi kita tidak bisa menolak distribusi
baju. Ini tidak membuktikan bahwa dadu kami adil. Tapi kita tidak bisa
menolak hipotesis kita.

Selanjutnya kita menghasilkan 1000 lemparan dadu menggunakan generator
angka acak, dan melakukan tes yang sama.
\end{eulercomment}
\begin{eulerprompt}
>n=1000; t=random([1,n*6]); chitest(count(t*6,6),dup(n,6)')
\end{eulerprompt}
\begin{euleroutput}
  0.594554930686
\end{euleroutput}
\begin{eulercomment}
Fungsi random([1,n*6]) digunakan untuk menghasilkan serangkaian
bilangan acak antara 1 dan 6 sebanyak n kali. Kemudian, fungsi
count(t*6,6) digunakan untuk menghitung berapa kali angka 6 muncul
dalam serangkaian bilangan acak tersebut. Fungsi dup(n,6) digunakan
untuk menghasilkan serangkaian bilangan 6 sebanyak n kali. Hasil dari
fungsi chitest(count(t*6,6),dup(n,6)) akan mengembalikan nilai p, yang
dapat digunakan untuk menentukan apakah serangkaian bilangan acak
tersebut terdistribusi secara merata atau tidak.

\end{eulercomment}
\eulersubheading{Uji T}
\begin{eulercomment}
Uji-t adalah sebuah metode statistik yang digunakan untuk menguji
perbedaan signifikan antara rata-rata dua kelompok yang berbeda.

Misalnya, akan dilakukan pengujian nilai rata-rata dari 100 elemen
data menggunakan uji-t.\\
Mari kita uji nilai rata-rata 100 dengan uji-t.
\end{eulercomment}
\begin{eulerprompt}
>s=200+normal([1,100])*10; ...
>ttest(mean(s),dev(s),100,200)
\end{eulerprompt}
\begin{euleroutput}
  0.161169176307
\end{euleroutput}
\begin{eulercomment}
Fungsi ttest() membutuhkan nilai rata-rata, simpangan, jumlah data,
dan nilai rata-rata yang akan diuji. Dalam syntax tersebut,
normal([1,100]) menghasilkan 100 angka acak yang diambil dari
distribusi normal dengan nilai rata-rata 0 dan simpangan baku 1.
Kemudian, nilai-nilai tersebut dikalikan dengan 10 dan ditambahkan
dengan 200 untuk menghasilkan sampel data dengan nilai rata-rata 200
dan simpangan baku 10.

Sekarang mari kita periksa dua pengukuran untuk mean yang sama. Kami
menolak hipotesis bahwa mereka memiliki rata-rata yang sama, jika
hasilnya \textless{}0,05.
\end{eulercomment}
\begin{eulerprompt}
>tcomparedata(normal(1,10),normal(1,10))
\end{eulerprompt}
\begin{euleroutput}
  0.23564235612
\end{euleroutput}
\begin{eulercomment}
Jika kita menambahkan bias ke satu distribusi, kita mendapatkan lebih
banyak penolakan. Ulangi simulasi ini beberapa kali untuk melihat
efeknya.
\end{eulercomment}
\begin{eulerprompt}
>tcomparedata(normal(1,10),normal(1,10)+2)
\end{eulerprompt}
\begin{euleroutput}
  0.000114481122387
\end{euleroutput}
\begin{eulercomment}
Pada contoh berikutnya, kita menghasilkan 20 lemparan dadu acak
sebanyak 100 kali dan menghitung yang ada di dalamnya. Harus ada
20/6=3,3 yang rata-rata.
\end{eulercomment}
\begin{eulerprompt}
>R=random(100,20); R=sum(R*6<=1)'; mean(R)
\end{eulerprompt}
\begin{euleroutput}
  3.31
\end{euleroutput}
\begin{eulercomment}
Bandingkan jumlah satu dengan distribusi binomial. Pertama lakukan
plot distribusi satuan.
\end{eulercomment}
\begin{eulerprompt}
>plot2d(R,distribution=max(R)+1,even=1,style="\(\backslash\)/"):
\end{eulerprompt}
\eulerimg{25}{images/Kelompok 6_EMT4Statistika-034.png}
\begin{eulerprompt}
>t=count(R,21);
\end{eulerprompt}
\begin{eulercomment}
Menghitung nilai yang diharapkan
\end{eulercomment}
\begin{eulerprompt}
>n=0:20; b=bin(20,n)*(1/6)^n*(5/6)^(20-n)*100;
\end{eulerprompt}
\begin{eulercomment}
Kita harus mengumpulkan beberapa angka untuk mendapatkan kategori yang
cukup besar.
\end{eulercomment}
\begin{eulerprompt}
>t1=sum(t[1:2])|t[3:7]|sum(t[8:21]); ...
>b1=sum(b[1:2])|b[3:7]|sum(b[8:21]);
\end{eulerprompt}
\begin{eulercomment}
Uji chi-kuadrat menolak hipotesis bahwa distribusi kami adalah
distribusi binomial, jika hasilnya \textless{}0,05.
\end{eulercomment}
\begin{eulerprompt}
>chitest(t1,b1)
\end{eulerprompt}
\begin{euleroutput}
  0.45755878393
\end{euleroutput}
\eulersubheading{Uji Independensi}
\begin{eulercomment}
Uji independensi adalah salah satu uji statistik yang digunakan untuk
mengetahui apakah ada hubungan antara dua variabel kategorik, dengan
kata lain untuk mengetahui independensi antara variabel baris dan
kolom. Uji ini berguna untuk mengukur perbedaan pengamatan dan
menaksir frekuensi suatu pengamatan dalam kategori tertentu.

Contoh berikut berisi hasil dua kelompok orang (laki-laki dan
perempuan, katakanlah) memberikan suara untuk 6 jenis minuman kaleng
\end{eulercomment}
\begin{eulerprompt}
>A=[23,37,43,52,64,74;27,39,41,49,63,76];  ...
>  writetable(A,wc=6,labr=["m","f"],labc=1:6)
\end{eulerprompt}
\begin{euleroutput}
             1     2     3     4     5     6
       m    23    37    43    52    64    74
       f    27    39    41    49    63    76
\end{euleroutput}
\begin{eulercomment}
Tes tabel chi\textasciicircum{}2 dapat melakukan uji independensi suara dari jenis
kelamin
\end{eulercomment}
\begin{eulerprompt}
>tabletest(A)
\end{eulerprompt}
\begin{euleroutput}
  0.990701632326
\end{euleroutput}
\begin{eulercomment}
Berikut ini adalah tabel yang diharapkan, jika kita mengasumsikan
frekuensi pemungutan suara yang diamati.
\end{eulercomment}
\begin{eulerprompt}
>writetable(expectedtable(A),wc=6,dc=1,labr=["m","f"],labc=1:6)
\end{eulerprompt}
\begin{euleroutput}
             1     2     3     4     5     6
       m  24.9  37.9  41.9  50.3  63.3  74.7
       f  25.1  38.1  42.1  50.7  63.7  75.3
\end{euleroutput}
\begin{eulercomment}
Kita dapat menghitung koefisien kontingensi yang dikoreksi. Karena
sangat dekat dengan 0, kami menyimpulkan bahwa pemilihan jenis minuman
kaleng tidak bergantung pada jenis kelamin.
\end{eulercomment}
\begin{eulerprompt}
>contingency(A)
\end{eulerprompt}
\begin{euleroutput}
  0.0427225484717
\end{euleroutput}
\begin{eulercomment}
\end{eulercomment}
\eulersubheading{Metode Anova}
\begin{eulercomment}
Selanjutnya kami menggunakan analisis varians (Uji-F) untuk menguji
tiga sampel data yang terdistribusi normal untuk nilai rata-rata yang
sama. Metode tersebut disebut ANOVA (analisis varians). Di Euler,
fungsi varanalysis() digunakan.
\end{eulercomment}
\begin{eulerprompt}
>x1=[109,111,98,119,91,118,109,99,115,109,94]; mean(x1)
\end{eulerprompt}
\begin{euleroutput}
  106.545454545
\end{euleroutput}
\begin{eulerprompt}
>x2=[120,124,115,139,114,110,113,120,117]; mean(x2),
\end{eulerprompt}
\begin{euleroutput}
  119.111111111
\end{euleroutput}
\begin{eulerprompt}
>x3=[120,112,115,110,105,134,105,130,121,111]; mean(x3)
\end{eulerprompt}
\begin{euleroutput}
  116.3
\end{euleroutput}
\begin{eulerprompt}
>varanalysis(x1,x2,x3)
\end{eulerprompt}
\begin{euleroutput}
  0.0138048221371
\end{euleroutput}
\begin{eulercomment}
Dalam pengujian hipotesis statistik di atas, hipotesis nol
mengasumsikan bahwa nilai rata-rata ke-3 set data (x1,x2,x3) sama.
Namun, berdasarkan analisis varians telah kita lakukan, hipotesis nol
dapat ditolak dengan probabilitas kesalahan 1,3\%. Ini berarti bahwa
ada perbedaan yang signifikan secara statistik antara rata-rata dari
ketiga set angka.
\end{eulercomment}
\begin{eulercomment}

\begin{eulercomment}
\eulerheading{Sub Topik 9 : Menyimpan Data Hasil Analisis}
\begin{eulercomment}
Data-data yang kita gunakan dalam melakukan analisis statistika dapat
kita simpan ke dalam suatu file sehingga ketika kelak ingin digunakan
lagi, data tersebut masih ada di file penyimpanan kita. Tak hanya itu,
hasil dari analisis statistika yang sudah kita lakukan pun dapat kita
simpan ke dalam suatu file.

Berikut adalah cara menyimpan/menulis data ke suatu file.
\end{eulercomment}
\begin{eulerprompt}
>a=random(1,100); mean(a); dev(a);
>filename="Simpan";
\end{eulerprompt}
\begin{eulercomment}
Seletah memberi nama untuk file, kita akan menulis vektor a ke dalam
file dengan menggunakan fungsi writematrix() dan menggunakan fungsi
readmatrix() untuk membaca data.
\end{eulercomment}
\begin{eulerprompt}
>writematrix(a',filename);
>a=readmatrix(filename)';
\end{eulerprompt}
\begin{eulercomment}
Kita juga bisa menghapus file yang sudah tersimpan dengan menggunakan
fileremove
\end{eulercomment}
\begin{eulerprompt}
>fileremove(filename);
\end{eulerprompt}
\begin{eulercomment}
Kemudian kita akan mencoba untuk menggantikan data baru ke file lama
dengan menghapus semua data lama, dan menulis lagi data baru yang akan
disimpan.
\end{eulercomment}
\begin{eulerprompt}
>file="Simpan"; open(file,"w");
>writeln("A,B,C"); writematrix(random(3,3));
>close();
>printfile(file)
\end{eulerprompt}
\begin{euleroutput}
  A,B,C
  0.8351051327697636,0.08458248162153686,0.5192250558799737
  0.8548977070796793,0.7679427770316303,0.4018472121828296
  0.02466356619093713,0.4253015574769268,0.1649367711598173
  
\end{euleroutput}
\begin{eulercomment}
Selain itu kita juga bisa menyimpan dalam bentuk excel
\end{eulercomment}
\begin{eulerprompt}
>file="test.csv";
>M=random(3,3); writematrix(M,file);
\end{eulerprompt}
\begin{eulercomment}
Berikut adalah isi dari file ini.
\end{eulercomment}
\begin{eulerprompt}
>printfile(file)
\end{eulerprompt}
\begin{euleroutput}
  0.2641349455581374,0.6713124187949838,0.135806558906826
  0.3437161954193733,0.7730785630232085,0.05730568239045469
  0.3519819858104311,0.1107098010416762,0.4220892269525604
  
\end{euleroutput}
\begin{eulercomment}
CVS ini dapat dibuka pada sistem bahasa Inggris ke dalam Excel dengan
klik dua kali. Jika Anda mendapatkan file seperti itu di sistem
Jerman, Anda perlu mengimpor data ke Excel dengan memperhatikan titik
desimal.

Tetapi titik desimal juga merupakan format default untuk EMT. Anda
dapat membaca matriks dari file dengan readmatrix().
\end{eulercomment}
\begin{eulerprompt}
>readmatrix(file)
\end{eulerprompt}
\begin{euleroutput}
       0.264135      0.671312      0.135807 
       0.343716      0.773079     0.0573057 
       0.351982       0.11071      0.422089 
\end{euleroutput}
\begin{euleroutput}
  
\end{euleroutput}
\end{eulernotebook}
\end{document}
